% Created 2020-12-27 Sun 23:59
% Intended LaTeX compiler: pdflatex
\documentclass[sigconf,natbib,screen=true,review=true,anonymous]{acmart}

% We'll get the submission number fro the submission system
\acmSubmissionID{848}

% !TEX root = ./main.tex

% list your packages
\usepackage[utf8]{inputenc}
\usepackage[T1]{fontenc}
\usepackage{graphicx}
\usepackage{grffile}
\usepackage{longtable}
\usepackage{wrapfig}
\usepackage{rotating}
\usepackage[normalem]{ulem}
%\usepackage{amsmath}
\usepackage{textcomp}
%\usepackage{amssymb}
\usepackage{capt-of}
\usepackage{hyperref}
%\usepackage{minted}

\usepackage{algorithmic}
\usepackage{graphicx}
\usepackage{xcolor}
\usepackage{multirow} 
\usepackage{xcolor}
\usepackage[ruled,linesnumbered,boxed]{algorithm2e}
\usepackage[inline]{enumitem}
\usepackage{booktabs}
\usepackage[justification=centering, skip=0pt]{caption}
\usepackage{acronym}
\usepackage{url}
% \usepackage{graphicx}
% \usepackage{adjustbox}

\usepackage{xcolor}

%\usepackage{amsmath}
\usepackage{dsfont}
\usepackage{color}

\usepackage{subcaption}
% !TEX root = ./main.tex

% list your definitions
\newcommand{\mdr}[1]{\textcolor{orange}{\textbf{[#1]}}}

\newcommand\todo[1]{\textcolor{red}{TODO : #1}}
\newcommand\doubt[1]{\textcolor{orange}{DOUBT : #1}}
% \newcommand\todo[1]{} % uncomment to hide comments
% \newcommand\doubt[1]{} % uncomment to hide comments

\hypersetup{
 pdfauthor={Gabriel Bénédict},
 pdftitle={},
 pdfkeywords={},
 pdfsubject={},
 pdfcreator={Emacs 26.1 (Org mode 9.1.14)}, 
 pdflang={English}}

% !TEX root = ./main.tex

\author{Gabriel B\'en\'edict}
\affiliation{
\institution{RTL \& University of Amsterdam} 
\city{Amsterdam}
\country{The Netherlands}
}
\email{something}

\author{Daan Odijk}
\affiliation{
\institution{RTL} 
\city{Amsterdam}
\country{The Netherlands}
}
\email{something}

\author{Maarten de Rijke}
\affiliation{
\instutute{University of Amsterdam \& Ahold Delhaize}
\city{Amsterdam}
\country{The Netherlands}
}
\email{m.derijke@uva.nl}

% !TEX root =  ./main.tex 

\copyrightyear{2021}
\acmYear{2021}
\setcopyright{rightsretained}
\acmConference[SIGIR '21]{Proceedings of the 44th International ACM SIGIR Conference on Research and Development in Information Retrieval}{July 11-15, 2021}{Montr{\'{e}}al, Canada}
\acmBooktitle{Proceedings of the 44th International ACM SIGIR Conference on Research and Development in Information Retrieval (SIGIR '21), July 11-15, 2021, Montr{\'{e}}al, Canada}
\acmPrice{}
\settopmatter{printfolios=false}


\begin{document}

\title[Confusion Matrix Metrics as Losses for Multilabel Classification]{Confusion Matrix Metrics as Losses for \\ Multilabel Classification with Unknown Label Counts}

  % SIMPUL: A Loss Framework to Learn Abstract Labels from Abstract Representations: Single-Instance Multiclass Multilabel Prediction with Unknown Label Count}

\begin{abstract}

Multiclass multilabel classification refers to the task of attributing labels to examples. A particular challenging setting is when both the number of labels for each example is unknown a priori and the example must be consumed in its entirety to be labeled. We refer to this problem setting as Full-Instance, Multilabel Prediction for Unknown Label count (FIMPUL). Current approaches to tackle FIMPUL problems tend to re-frame the problem as a multiclass unilabel classification problem. Two commonly used approaches are (i) optimizing for variations of the traditional cross-entropy loss (the fit-data-to-algorithm approach), or (ii) adapt existing algorithms to the problem at hand (the fit-algorithm-to-data approach).

We take a design-algorithm-for-data approach tailored specifically for the FIMPUL problem and propose a new framework (CoMMaL) for loss functions based on confusion matrix metrics. With CoMMaL, we propose smooth surrogates to confusion matrix metrics that are decomposable for optimization methods such as Stochastic Gradient Descent. Within the CoMMaL framework, we specifically evaluate the effectiveness of sigmoidF1, a smooth surrogate F1 loss, on both text and image data. We embed sigmoidF1 in a classification head that is attached to state-of-the-art pretrained neural networks MobileNetV2 and DistilBERT.

Our experiments show that sigmoidF1 outperforms other existing loss functions in four datasets on several metrics. These results show the effectiveness of using traditional inference-time metrics as loss function at training time.

% \mdr{Multilabel classification is the task of classifying XXX.}
% \mdr{In multilabel classification with unknown label counts we do not know XXX.}
% \mdr{Multilabel classification with unknown label counts are very common in IR.}
% \mdr{Current approaches to Multilabel classification with unknown label counts are characterized by XXX and suffer from YYY.}
% \mdr{We propose XXX to address Multilabel classification with unknown label counts.}
% \mdr{We evaluate our proposal on XXX.}
% \mdr{We find XXX.}
% \mdr{Broader implications.}
\if0
Multilabel classification is a common task when learning from text, image, or sound. However, few optimization frameworks are tailored towards learning multiple abstract labels for an entire text, image, or sound. State of the art models for image and text, of the CNN and BERT family respectively, are often benchmarked on multiclass multilabel datasets, but still use loss functions adapted to a multiclass unilabel situation (i.e. variations of cross-entropy losses).

When the number of ground-truth labels varies over each example as is commonly the case in multilabel classification, these losses produce unit-interval results that require a sophisticated thresholding regime (at training or at inference time) to predict both label prediction propensity and label count.

Since certain confusion matrix metrics usually already fulfill that goal at inference time, we propose decomposable surrogates for gradient descent at training time. We illustrate the solution with \emph{sigmoidF1}: a decomposable surrogate F1 score that introduces smooth thresholding. We are able to demonstrate its performance on efficient versions of the state of the art models (DistilBert and MobileNetV2). For a fair assessment of the proposed loss function, we avoid datasets that would benefit from a two stage modeling procedure (object / expression / segment detection followed by classification) and instead turn to datasets, where the whole example is predictive of a label.
\fi
\end{abstract}


\maketitle

\acresetall

% !TEX root = ../main.tex

\section{introduction}
\label{sec:org662677c}

As neural network models are able to learn increasingly more abstract representations via deeper networks, representation learning and self-supervision~\citep[see, e.g.,][]{XX,YYY,ZZZ}, it is reasonable to expect that they also get better at predicting more abstract labels. 
Beyond identifying object types~\mdr{REF}, performing face recognition~\mdr{REF}, \mdr{doing what with} expressions~\mdr{REF}, neural networks should soon be able to predict genres/categories of text, image and sound at high levels of accuracy. 
Towards this goal, there is a significant volume of recent work on building neural networks with high-level of understanding in the embedding space~\mdr{REF}.
However, there seems to be limited research on developing loss functions that are adapted for these higher level concepts in the output space.

\mdr{Briefly and concretely describe the abstract labeling task that we are interested in.}

To situate the specific problem that we tackle in this paper, it is helpful to consider Figure~\ref{fig:tree} so as to disambiguate our terminology. 
Figure~\ref{fig:tree} shows \mdr{what}.
There seems to exist a consensus over the terms \emph{multiclass} and \emph{multilabel learning}, meaning mutually exclusive and mutually inclusive labels, respectively \todo{source}. 
Multilabel learning can therefore be seen as a subdomain of multiclass learning, where more than one class can be true for the same example. 
Within multilabel training, we introduce the distinction between multi-instance multilabel~\citep[e.g.,][]{multiInstance}) and uni-instance multilabel. 
\mdr{vague: Multi-instance multilabel classification refers to tasks where elements within each example can be singled-out and assigned one or more labels; examples include objects \mdr{not a task} in an image or tokens in a text.}
Uni-stance multilabel classification is \mdr{define it}.
\begin{itemize}
\item \mdr{Now explain that the task introduced in the second paragraph is an instance of uni-instance multilabel classification.}
\item \mdr{Now make the distinction between fixed label counts and varying label counts}
\item \mdr{Little work has been done on the uni-instance, ML MC prediction with a varying number of classes}
\item \mdr{Identify the shortcomings o prior work on the task that we care about}
\end{itemize}

\begin{figure}[t]
\centering
\includegraphics[width=.9\linewidth]{./tree/Tree.pdf}
\caption{\label{fig:tree}
Clarifying ``multiclass'' classification problems.
In this paper we focus on the uni-instance, multilabel, multiclass classification problem with a varying number of labels (the bottom right hand side of the tree).
\mdr{Image source ...}}
\end{figure}

\mdr{Now we have a paragraph in which you clearly describe your proposed line of attack}

\mdr{Now we have a paragraph that explains the results that we have obtained with our proposed approach}

\mdr{Now we have a paragraph with our main contributions:}
\begin{itemize}[leftmargin=*]
\item \mdr{Contribution 1}
\item \mdr{Contribution 2}
\item \mdr{Contribution 3}
\end{itemize}

\mdr{The remainder of the paper is organized as follows. Expand.}

\vspace*{3cm}
\mdr{Move all of the content to other sections, e.g., to background section or to related work. Also, try to avoid the meandering narrative that touches on many points but sometimes forgets to make explicit what its main point is.}

In this paper we focus on uni-instance multilabel training (sparse occurences of the term holistic can be found in the literature to describe this phenomenon for image \cite{holisticImageDescriptors,holisticLungs} and a recent video dataset \cite{holisticVideoData} \todo{read these}), more specifically with varying label counts. To the best of our knowledge, there are few existing representatives of that type of labelling task in the literature. \todo{cite more milestone examples for each category.} \todo{delta with hierarchical label learning}

\mdr{Too talkative, too many diverse angles. Make sure that there is a clear point that you are making:}
Although multilabel binary prediction (commonly referring to mutually inclusive labels) is a task thoroughly covered in existing literature, there does not seem to exist a framework that deals with different amounts of positive labels in the groundtruth. For example, a scientific journal can be tagged as \emph{machine learning} and \emph{economics}, or a movie can be tagged as \emph{romance} and \emph{comedy}. These instances might as well be assigned only one tag in the groundtruth, or many more within the possible tags (classes).


The particularity of tasks like scientific paper tagging or movie genre classification is that it remains unclear what elements in an image/video or text can be singled out as predictive of a particular tag/genre. Rather, a complex interaction between these elements in the feature space steer the predictions. For example, the sole mention of the term "machine learning" in a paper should not be a sufficient condition to tag it as such. Instead, one could expect from the publisher to get acquainted with the paper enough to determine wether the research is a worthwhile contribution or application of \emph{machine learning} to deserve the tag. This involves thorough understanding of the proposed method and background knowledge on state-of-the-art methods. An analogous argument can be made for movie genre classification for movie posters.

However, if elements in an image/text can be singled out as predictive of a single tag, the problem reverts back to predicting with the a priori knowledge of the existence of only one true label (i.e. multi-instance multilabel learning).  The reason for distanciating singling-out from uni-instance labels, is that it has been shown that as soon as singling-out is possible, models that work on instances are more accurate \todo{rewrite this paragraph and sources}. The singled-out elements can be subsets of the original feature space (typically in object detection like with the COCO dataset  \cite{COCO} or the Amazon Rainforest Dataset\footnote{Available at \url{https://www.kaggle.com/c/planet-understanding-the-amazon-from-space}} \todo{others}). Similarly, recent research has shown that the singled-out elements can be located in the abstract representations (embeddings) of the feature set and might individually predict a single true label (like GPT-3 \todo{source}) \todo{more examples}. This might also carry prospects of generalizability of the model \cite{generalization} \todo{elaborate}. 

But for now, in certain retrieval tasks such as scientific journal tagging, the effect of sub-entities (either expressions in the text or single features in the embedding space) on the prediction of each label remains hard to assess. Instead we propose uni-instance (sometimes referred to as holistic) multilabel learning for varying amount of labels, with a focus on custom loss functions.

To allow the use of existing diffentiable loss fonctions, previous research papers tend to reframe the problem into either (I) a multi-instance multiclass (as described above, with the COCO dataset as an example of isolation of features \cite{COCO}), (II) uni-instance multiclass prediction (III) uni-instance multilabel prediction with fixed label count (IV) uni-instance multilabel prediction with varying label count with post-training thresholding (V) redefine backpropagation for multilabel prediction \cite{multilabelBackprop} (VI) multitask learning \cite{multitaskLabel} (VII) custom loss function \cite{tencent}. This order reflects in ascending order how close modelling seem to fit the original task, which remains uni-instance multilabel learning with varying amounts of labels. \doubt{group them}

Common loss functions such as cross-entropy loss (for mutually inclusive labels) or multinomial logit loss (for mutually exclusive labels) deliver predictions on the unit interval. Thresholding the output to assess the performance of the model against the groundtruth can be done after training for (I), (II), (III) and (IV). \todo{give a very sound reason as to why we'd rather not do things post-training and rather at training-time}. Problem formulations (V), (VI) and (VII) suggest a solution at training time. We think that a custom loss function (VII) is the best alternative. \todo{explain why}

In a number of retrieval tasks, a model's out of sample accuracy is measured on metrics such as AUROC, F1 score, etc. These reflect an objective catered towards evaluating the model over an entire ranking. Due to to lack of differentiability, these metrics cannot be directly used as loss functions at training time (in-sample). A seminal study \cite{optimizableLosses} derived a general framework for deriving decomposable surrogates to some of these metrics. We propose our own decomposable F1 surrogate tailored for the problem at hand.

\mdr{This paragraph can probably be kept in the introduction}
We first propose a general mathematical formulation of uni-instance multilabel learning for varying amount of groundtruth labels. The generalization encompasses different levels of complexity, from the classical cross-entropy loss up to the proposed loss function. \emph{sigmoidF1} is a F1 score surrogate which allows to optimize for label prediction and count simultanuously in a single task and is robust to outliers. It delivers more precise predictions than the current state-of-the-art on several different metrics, accross text and image related tasks. \emph{sigmoidF1} and its adaptive \emph{SadF1} and Bayesian \emph{SBayesF1} counterparts are benchmarked against loss functions commonly used in multilabel learning and others tailored specifically to the uni-instance multilabel with varying number of labels setting.

% !TEX root = ../main.tex

\section{Background}
\label{section:background}

To build up to our approach, we use a traditional statistical framework as a guideline for multilabel classificaton methods~\citep{tukey}. We distinguish the desired theoretical statistic (the \textbf{estimand}), its functional form (the \textbf{estimator}) and its approximation (the \textbf{estimate}); the latter estimates can be benchmarked with \textbf{metrics}. We show how multilabel reductions tend to reformulate the estimand and treat labels independently (i.e. change our assumptions about the data). However, with a proper estimator, it is possible to directly model the estimand. Our proposed loss function, \textbf{sigmoidF1}, accommodates for the true estimand.

We define a learning algorithm $\mathcal{F}$ (i.e. a class of estimators) that maps inputs to outputs given a set of hyperparameters \(\mathcal{F}(\cdot ; \Theta): \mathcal{X} \rightarrow \mathcal{Y}\). We consider a particular case, with the input vector \(\mathbf{x} = \{x_1, \ldots, x_n\}\) and each observation is assigned $k$ labels (one or more) \(\mathbf{l} = \{l_1, \ldots, l_C\}\) out of a set of $C$ classes. \(y_{i}^{j}\) are binary variables, indicating presence of a label for each observation \(i\) and class \(j\). Together they form the matrix output $Y$.

\subsection{Estimand and definition of the risk}
\label{section:background:estimand}

We distinguish between two scenarios: the \emph{multiclass} and the \emph{multilabel} scenario. 
In the multiclass scenario, a single example is attributed one class label (e.g., classification of an animal on a picture). 
In the multilabel scenario, a single example can be assigned more than one class label (e.g., movie genres). 
%Multiclass classification deserves a mention towards the end, as it illustrates a straightforward relationship between estimand-estimator-estimate. 
We focus on the multilabel scenario. More formally, for a particular set of input $\mathbf{x}$ (e.g. paper abstracts) and output $Y$ (e.g. scientific field(s) ) its risk formulation can be written as in ~\citep{multilabelReduction}:
%
\begin{equation}
R_{\mathrm{ML}}(\mathcal{F}) = \mathbb{E}_{(\mathbf{x}, Y)}\left[\mathcal{L}_{\mathrm{ML}}(Y, \mathcal{F}(\mathbf{x}))\right].
\end{equation}
%
We define $\mathcal{F}$ as the estimand: the theoretical statistic. For one item $x_i$, the theoretical risk defines how close the estimand can get to that deterministic output vector $\mathbf{y}_{i}$.

\if0
For example, corn ($x_{corn}$) is eaten by a finite number of animals that can digest it ($\mathbf{y}_{corn} = {y_{corn}^{horse}, \ldots, y_{corn}^{sheep}} $), assuming animals belong to a finite set of living beings. If one were to predict which food is eaten by which animals, the theoretical risk (estimand) defines how close we can get to that deterministic vector $\mathbf{y}_{corn}$, given $x_{corn}$.
\fi

In practice, statistical models do output probabilities $\mathbf{\hat{y}}_{i}$ via an estimator and its estimate (also called propensities or suitabilities~\citep{multilabelReduction}). The solution to that stochastic-deterministic incompatibility is either to convert the estimator to a deterministic measure via decision thresholds (e.g. traditional cross-entropy loss), or to treat the estimand as a stochastic measure (our sigmoidF1 loss proposal).

\subsection{Estimator: the functional form}
\label{section:background:estimator}

The estimator $f \in \mathcal{F}$ is any minimizer of the risk $R_{ML}$. Predicting multiple labels per example comes with the assumption that labels are non-mutually exclusive.

\vspace{-.5\baselineskip}
\begin{proposition}
  The multilabel estimator of $y_{i}^{j}$ is dependent on the input and other labels for that example,
%
\begin{equation}
  \hat{y}_i^j = f(x, y_{i}^{1}, \ldots, y_{i}^{j-1}) = P(y_i^j = 1 | x, y_{i}^{1}, \ldots, y_{i}^{j-1})
\end{equation}
\label{eq:estimator}
\end{proposition}
\vspace{-1.5\baselineskip}
%
 By proposing this general formulation, we entrench that characteristic in the estimator. Contrary to \citet{multilabelReduction}, we propose an estimator that models interdependence between labels and deals with thresholding for the estimate at training time for free.


\subsection{Estimate: approximation via a loss function}
\label{section:background:estimate}

Most of the literature found on multilabel classification can be characterized as multilabel reductions~\cite{multilabelReduction}. Given the general non-convex optimization context, the surrogate loss function $\mathcal{L}(\mathbf{y}_i, f)$ can take different forms\footnote{Note that OVA and PAL have each a form normalised by the number of positive labels~\cite{multilabelReduction}. We leave out pick-one-label, as it is further removed from our discussion.}. 

\subsubsection*{One-versus-all (OVA)}
OVA consists of a reformulation of the multilabel classification task to a sequence of $C$ binary classifications ($f^1, \ldots, f^C$), with $C$ the number of classes:
%
\begin{equation}
\begin{aligned}
\mathcal{L}_{\mathrm{OVA}}(\mathbf{y}_i, f) &= \sum_{c = 1}^C \mathcal{L}_{\mathrm{BC}}\left(y_i^{c}, f^{c}\right)\\
&=\sum_{c = 1}^C \left\{y_{i}^c \cdot \mathcal{L}_{\mathrm{BC}}\left(1, f^c \right)+\left(1-y_{i}^c \right) \cdot \mathcal{L}_{\mathrm{BC}}\left(0, f^c \right)\right\}
\end{aligned}
\end{equation}
%
where $\mathcal{L}_{\mathrm{BC}}$ is a binary classification loss (binary relevance \cite{OVA1, hammingLoss, OVA2}), most often logistic loss.  Note that minimizing binary cross-entropy is equivalent to maximizing for log-likelihood~\cite[\S4.3.4]{Bishop}.

\subsubsection*{Pick-all-labels (PAL)}
The loss function set here is
%
\vspace{-0.5\baselineskip}
\begin{equation}
\mathcal{L}_{\mathrm{PAL}}(\mathbf{y}_i, f) = \sum_{c = 1}^C y_{i}^c \cdot \mathcal{L}_{\mathrm{MC}}(y_i^c, f),
\end{equation}
%
with $\mathcal{L}_{MC} : C \times \mathbb{R}^{C} \rightarrow \mathbb{R}_{+}$, a multiclass loss (e.g. softmax cross-entropy). In this formulation, each example $(x_i, \mathbf{y}_i)$ is converted to a multiclass framework, with one observation per positive label. The sum of inherently multiclass losses is used to represent the multilabel estimand. Note that cross-entropy loss can be formulated as \(\mathcal{L}_{\text {CE}}=-\sum \log \left(f \right)\).

\vspace{\baselineskip}

Multilabel reduction methods are characterized by their way of reformulating the estimand, the resulting estimator and the estimate. This allows the use of existing losses: logistic loss (for binary classification formulations), sigmoid loss or softmax cross-entropy loss (for multiclass formulations). These reductions imply a reformulation of the estimator (a.k.a. Bayes Optimal) as follows:
%
\begin{equation}
  \hat{y}_i^j = f(x) = P(y_i^j = 1 | x_i)
\end{equation}
%
Contrary to our definition of the original multilabel estimator (Eq.~\ref{eq:estimator}), independence of label propensities is assumed. In other words the loss function becomes any monotone transformation of the marginal label probabilities $P(y_i^j = 1 | x)$ ~\cite{OVA2, multilabelMetrics, unifiedView}.

% \citet{multilabelReduction} show that OVA, PAL and their normalized counterparts are consistent~\citep{consistency-surrogates, consistency-multiclassSVM, consistency-lossAnalysis} with either precision or recall. A learning algorithm is defined as consistent if the expected risk given $f$ and $\mathcal{L}$ approximates the original estimator (Bayes risk), as the input sample size grows~\cite{consistencyMultilabel}.

\subsection{Metrics: evaluation at inference time}
\label{section:background:metrics}

There is a consensus around the use of confusion matrix  and ranking metrics to evaluate multilabel classification models (at inference time)~\cite{multilabelMetrics, weightedMetrics, unifiedView}. Notably Precision and Recall. Confusion matrix metrics come with caveats: most of these measures 
\begin{enumerate*}
\item require a hard thresholding, and that makes them non-differentiable for Stochastic Gadiant Descent, 
\item they are very sensitive to the choice of the number top labels to include $k$\footnote{In the case of unilabel prediction, top-k becomes a top-1 problem, which essentially eliminates caveats I and II.} and 
\item they require aggregation choices to be made in terms of micro / macro / weighted metrics.
\end{enumerate*}
Some common confusion matrix metrics are Precision, Recall, F1-score, hinge-loss or one-error-loss. Numerous others can be formulated~\cite{unifiedView}.


\subsection{Multilabel estimate: F1 Metric as a Loss}
\label{section:background:metricsAsLosses}


In a number of retrieval tasks, a model's out of sample accuracy is measured on metrics such as AUROC, F1 score, etc. These reflect an objective catered towards evaluating the model over an entire ranking. Due to the lack of differentiability, these metrics cannot be directly used as loss functions at training time (in-sample). A seminal Study~\cite{optimizableLosses} derived a general framework for deriving decomposable surrogates to some of these metrics. We propose our own decomposable surrogates tailored for the problem at hand.

In a typical machine learning classification task, binary labels are compared to a probabilistic measure (or a reversible
transformation of a probabilistic measure such as a sigmoid or a softmax
function). If the number $n_i$ of labels to be predicted per
example is known a priori, it is natural at training time to assign the $top_{n_i}$ predictions
to that example~\cite{lossTopKError, topKmulticlassSVM}. If the number of
labels per example is not known a priori, the question remains at both training and at inference time
as to how to decide on the number of labels to assign to each
example. This is generally done via a \emph{decision threshold}, that can be set globally for all
examples. This threshold can optimize for specificity or
sensitivity~\cite{decisionThreshold}. We propose an approach where this threshold is implicitly defined, by using a loss function that penalizes explicitly for wrong label counts and fits to the original estimand in Proposition 1.

That loss function is a surrogate of the F1 score, the harmonic mean of precision and recall (formal definition below). It implicitly deals with label counts and label predictions by including confusion matrix count data. It also balances precision and recall. In the next section, we show how $F_1$ is formulated into a surrogate loss $\mathcal{L}_{\widetilde{\mathit{F1}}}$.



% \begin{figure}[t]
% \centering
% \includegraphics[width=.9\linewidth]{./tree/Tree.pdf}
% \caption{\label{fig:tree} SIMPUL (bold) within the \emph{multiclass}
% nomenclature
% \hvk{figure is not referenced in text, bold is unclear in figure}
% \daan{I think it should go. And if it stays: uni -> single.}
% % Clarifying ``multiclass'' classification problems. In this paper we focus on
% % the uni-instance, multilabel, multiclass classification problem with a
% % varying number of labels (the bottom right hand side of the tree).
% }
% \end{figure}% \mdr{Image source ...}

%%% Local Variables:
%%% mode: latex
%%% TeX-master: "../main"
%%% End:
% !TEX root = ../main.tex

\section{Method}
\label{sec:orga8a42f5}
\label{section:method}

In this section, we introduce the smooth confusion matrix and in particular sigmoidF1. Confusion matrix metrics are well suited for FIMPUL problems because they implicitly deal with label count.

For the same class of learning algorithms defined in the previous section \(\mathcal{F}(\cdot ; \Theta): \mathcal{X} \rightarrow \mathcal{Y}\), we consider a particular case, where \(\mathbf{x} = \{x_1, \ldots, x_n\}\). Each observation is assigned one or more classes out of a set of labels \(\mathbf{l} = \{l_1, \ldots, l_c\}\). \(y_{i}^{j}\) are binary variables, indicating presence of a label for each observation \(i\) and class \(j\).

% For each observation \(i\), label class probabilities can be defined based on predictions as

% \todo{check this formula}

% \begin{equation}
% \mathbf{p}_{i}=\left\{\begin{array}{ll}\hat{\mathbf{y}} & \text { if } y=1 \\ 1-\hat{\mathbf{y}} & \text { otherwise }\end{array}\right.
% \end{equation}

Let \(tp\), \(fp\), \(fn\), \(tn\) be number of true positives, false positives, false negatives and true negatives respectively. It is necessary to define a threshold \(t\), at which a prediction is dichotomized:
%
\begin{equation}
\label{eq:conf}
\begin{array}{ll}\mathit{tp} = \sum \mathds{1}_{\hat{\mathbf{y}} \geq t} \odot \mathbf{y}  & \mathit{fp} = \sum \mathds{1}_{\hat{\mathbf{y}} \geq t} \odot (\mathds{1} - \mathbf{y}) \\[.5em] \mathit{fn} = \sum \mathds{1}_{\hat{\mathbf{y}} < t} \odot \mathbf{y} & \mathit{tn} = \sum \mathds{1}_{\hat{\mathbf{y}} < t} \odot (\mathds{1} - \mathbf{y}),
\end{array}
\end{equation}
%
%, \(\sum^+\) and \(\sum^-\) corresponding to  \(\sum_{i \in Y^{+}}\) and \(\sum_{i \in Y^{-}}\) respectively
with \(\odot\) the component-wise multiplication sign and \(\mathds{1}\) an indicator function. \(\sum \mathds{1}_{\hat{\mathbf{y}} \geq t}\), \(\sum \mathds{1}_{\hat{\mathbf{y}} < t}\) are thus the count of positive and negative predictions at threshold \(t\).

Note that in the formulation above and the ones that follow, scores are calculated for a single class, therefore the sum is implicit over all examples \(\sum_i\). This is useful for the binary classification problem but also for the multilabel problem, when micro metrics are calculated (i.e. metric for each class which is then averaged over all classes. At the end of this section we further refine the micro and macro concepts \hvk{this is unclear to me. Also, where is this?}).

In the multilabel setting $\mathbf{y}$ can be substituted by $\mathbf{y}^j$ for each class $j$. Note that vectors could be trivially substituted by matrices ($Y$) in the following expressions to obtain the macro formulation.
\doubt{do you agree that the subscript for the sum is implicit? and do you agree that the matrix formulation is trivial?}\hvk{do we really need this paragraph?}
\daan{I think we need a confusion matrix primer at the start of 3. We can then succinctly discuss multiclass predictions and might already cover a bit of micro vs macro. Or we can pick that up here.}

% Note that Equation \ref{eq:conf} represents a macro measure. For example, the micro $tp$ measure (i.e. for each class) would be $\sum \mathds{1}_{\hat{\mathbf{y}} \geq b} \odot \mathbf{y}$ $\sum_{i \in \hat{\mathbf{y}}^{j+}} \mathds{1}_{\hat{y}^{j}_{i} \geq b} \odot y^{j}_{i}$ (see end of this section, where we further refine the micro and macro concepts).

Given the four confusion matrix quadrants, we can generate further metrics like precision and recall (see Table \ref{tab:confusion-matrix}). However none of these metrics are decomposable due to the hard thresholding which is in effect a step function (see Figure \ref{fig:sigmoid}).

In the following sections, we first define desirable properties for decomposable thresholding.
Next, we define unbounded confusion matrix entries and a notion of sigmoid-based transformation that renders confusion matrix entries decomposable. Finally, we focus on an unbounded F1 score wich we use in our experiments.

\subsection{Desirable Properties of Decomposable Thresholding}

We thus define desirable properties for a decomposable sign function $f(u)$ as a surrogate of the above indicator function \(\mathds{1}_{\hat{\mathbf{y}} < t}\).

\begin{prop}
  Boundedness: $|f(u)| < M$, where $M$ is a an upper and lower bound.
\end{prop}
The groundtruth $\mathbf{y}$ is bounded between $[0,1]$ and thus it must be compared to a bounded prediction $\mathbf{\hat{y}}$, preferably bounded by $[0,1]$, to avoid further scaling.

\begin{prop}
  Saturation: $\int_{s}^{\infty} f(u) = \int_{-\infty}^{-s} f(u) = \epsilon$, with $\epsilon$ a number close to zero and $s$ a saturation bound.
\end{prop}
For the surrogate to be a proper sign function substitute, it is important to often return values close to 1 or 0. Saturation is defined in the context of neural network activation functions and refers to the propensity of iterative backpropagation to progressively lead to values very close to 0 or 1 after a long enough training period. While activation functions should tend to be non-saturated, in order for the derivative at point $u$ to be non-null and information to flow back to the network~\cite{saturation}, our sign function substitute must output values close to 0 or 1, in order to be comparable to a step function.

\begin{prop}
  Dynamic Gradient: $f'(u) >> 0 \quad \forall \; u \in [-s, s]$, with $s$ the saturation bound
\end{prop}

Outside the saturation bounds $[-s, s]$, the derivative should be significantly higher than zero in order to facilitate stochastic gradient descent and backpropagation.

Note that the upper and lower limits of $f(u)$ are sometimes $[-1,1]$ instead of $[0,1]$ along this paper. It is easy to show that the desirable properties above are still valid. \hvk{if its easy, this needs to be shown here.}

\subsection{Unbounded Alternative}
\label{subsec:unbounded}

A first trivial remedy to allow for derivation of the sign function, is to define \emph{unbounded} confusion matrix entries by replacing the dichotomized predictions with prediction probabilities. This way,
 (i.e. \(\overline{tp}\), \(\overline{fp}\), \(\overline{fn}\) and  \(\overline{tn}\) are not natural numbers anymore):

\begin{equation}
\label{eq:unbounded}
\begin{array}{ll} \overline{\mathit{tp}} = \sum \hat{\mathbf{y}} \odot \mathbf{y}  & \overline{\mathit{fp}} = \sum \hat{\mathbf{y}} \odot (\mathds{1} - \mathbf{y}) \\[.5em] \overline{\mathit{fn}} = \sum (\mathds{1} - \hat{\mathbf{y}}) \odot \mathbf{y} & \overline{\mathit{tn}} = \sum (\mathds{1} - \hat{\mathbf{y}}) \odot (\mathds{1} - \mathbf{y}),
\end{array}
\end{equation}

% $$
% \overline{tp}=\sum \hat{\mathbf{y}} \odot \mathbf{y} \quad \overline{fp} = \sum \hat{\mathbf{y}} \odot (\mathbf{1}- \mathbf{y}) \quad \overline{fn} = \sum (\mathbf{1} - \hat{\mathbf{y}}) \odot \mathbf{y}
% $$

\(tp\), \(fp\), \(fn\) and \(tn\) are now replaced by rough surrogates. The disadvantages are that the desirable properties mentioned above are not fulfilled, namely (i) \(\hat{\mathbf{y}}\) is unbounded and thus certain examples can have over-proportional effects on the loss, (ii) It is non-saturated. While non-saturation is desirable for activation functions~\cite{saturation}, it would be here desirable to tend towards saturation (i.e. tend to values close to 0 or 1, so as to give the most accurate predictions at any thresholding values at inference time). (iii) The gradient of that linear function is 1 and therefore backpropagation will not learn depending on different inputs at this stage of the loss function.

However, this method has the advantage of resulting in a linear loss function that avoids the concept of thresholding altogether and is trivial to decompose for stochastic gradient descent.

\subsection{Smooth confusion matrix entries}

We propose a sigmoid-based transformation of the confusion matrix that renders its entries decomposable and fulfills the three properties above:

\begin{equation}
\label{eq:smooth}
\begin{array}{ll} \widetilde{\mathit{tp}} = \sum \mathbf{S}(\hat{\mathbf{y}}) \odot \mathbf{y}  & \widetilde{\mathit{fp}} = \sum \mathbf{S}(\hat{\mathbf{y}}) \odot (\mathds{1} - \mathbf{y}) \\[.5em] \widetilde{\mathit{fn}} = \sum (\mathds{1} - \mathbf{S}(\hat{\mathbf{y}})) \odot \mathbf{y} & \widetilde{\mathit{tn}} = \sum (\mathds{1} - \mathbf{S}(\hat{\mathbf{y}})) \odot (\mathds{1} - \mathbf{y}),
\end{array}
\end{equation}

with $\mathbf{S}(\cdot)$ the vectorial form of the sigmoid function $S(\cdot)$:

\begin{equation}
S(u; \beta, \eta)=\frac{1}{1+\exp (-\beta (u + \eta))},
\end{equation}

with \(\beta\) and \(\eta\) tunable parameters for slope and offset respectively. Higher \(\beta\) results in steeper slope at the center of the sigmoid and thus more stringent thresholding. At its extreme, \(lim_{\beta\to\infty} S(u; \beta, \eta)\) corresponds to the step function used in Equation \ref{eq:conf}. Note that negative values of $\beta$ geometrically reflect the sigmoid function across the horizontal line at $0.5$ and thus invert predictions.


These smooth confusion matrix entries allow to build any related metric (see Table \ref{tab:confusion-matrix}). Furthermore, the surrogate entries are decomposable, bounded, saturated and have a dynamic gradient.

\begin{table*}[]
\caption{Confusion matrix with our proposed smoothed confusion matrix entries, $\widetilde{\mathit{tp}}$, $\widetilde{\mathit{fp}}$, $\widetilde{\mathit{fn}}$ and $\widetilde{\mathit{tn}}$ and six derived loss functions that use these smoothed confusion matrix entries, with $\mathcal{L}_{\widetilde{\mathit{F1}}}$ that is used in our experiments.}
\label{tab:confusion-matrix}
\def\arraystretch{1.1}
\begin{tabular}{|c||c|c||c|} \cline{2-4}
\multicolumn{1}{l|}{} & \textbf{Condition} & \textbf{Condition} & \multirow{2}{*}{$\mathcal{L}_{\widetilde{\mathit{Accuracy}}}= \frac{\widetilde{\mathit{tp}} + \widetilde{\mathit{tn}}}{\widetilde{\mathit{tp}} + \widetilde{\mathit{fp}} + \widetilde{\mathit{tn}} + \widetilde{\mathit{fn}}}$} \\
\multicolumn{1}{l|}{} & \textbf{positive} &  \textbf{negative} & \\ \hline \hline
\textbf{~Predicted~} & True positive & False positive & \multirow{2}{*}{$\mathcal{L}_{\widetilde{\mathit{Precision}}}= \frac{\widetilde{\mathit{tp}}}{\widetilde{\mathit{tp}} + \widetilde{\mathit{fp}}}$} \\
\textbf{positive} & $\widetilde{\mathit{tp}}=\sum \mathbf{S}(\hat{\mathbf{y}}) \odot \mathbf{y}$ & $\widetilde{\mathit{fp}}= \sum \mathbf{S}(\hat{\mathbf{y}}) \odot (\mathds{1} - \mathbf{y})$ & \\ \hline
\textbf{Predicted} & False negative & True Negative & \multirow{2}{*}{$\mathcal{L}_{\widetilde{\mathit{NPV}}}= \frac{\widetilde{\mathit{tn}}}{\widetilde{\mathit{tn}} + \widetilde{\mathit{fn}}}$} \\
\textbf{negative} & $\widetilde{\mathit{fn}}= \sum (\mathds{1} - \mathbf{S}(\hat{\mathbf{y}})) \odot \mathbf{y}$ & $\widetilde{\mathit{tn}}= \sum (\mathds{1} - \mathbf{S}(\hat{\mathbf{y}})) \odot (\mathds{1} - \mathbf{y})$ & \\ \hline \hline
\multicolumn{1}{l|}{} & \multirow{2}{*}{\hspace{1.2em}$\mathcal{L}_{\widetilde{\mathit{Recall}}}= \frac{\widetilde{\mathit{tp}}}{\widetilde{\mathit{tp}} + \widetilde{\mathit{fn}}}$\hspace{1.2em}}& \multirow{2}{*}{$\mathcal{L}_{\widetilde{\mathit{Specificity}}}= \frac{\widetilde{\mathit{tn}}}{\widetilde{\mathit{fp}} + \widetilde{\mathit{tn}}}$} & \multirow{2}{*}{$\mathcal{L}_{\widetilde{\mathit{F1}}}= \frac{\widetilde{\mathit{tp}}}{2 \widetilde{\mathit{tp}}+ \widetilde{\mathit{fn}}+ \widetilde{\mathit{fp}}}$} \\
\multicolumn{1}{l|}{} & & & \\
\cline{2-4}
\end{tabular}%
\end{table*}

In this paper we focus on $\mathcal{L}_{\widetilde{\mathit{F1}}}$ because it has the ability to implicitly penalize for both inacurate label propensity and label count.


\subsection{sigmoidF1 score}
\label{sec:orgc5d29d7}

Following Equation \ref{eq:unbounded}, it is possible to define an \emph{unbounded F1} score (see~\cite{softF1} for a similar formulation):

\begin{equation}
\mathcal{L}_{\overline{\mathit{F1}}}= \frac{\overline{tp}}{2 \overline{tp}+ \overline{fn}+ \overline{fp}}
\end{equation}


unboundedF1 inherits from the properties listed in Section~\ref{subsec:unbounded}. This score requires a high enough number of representatives in the four confusion matrix quadrants to learn from batch to batch. Ideally, each training epoch would have only one batch, so as to have the most representatives.

unboundedF1 and sigmoidF1 below are thought of as macro scores (aggregated over all classes). Given enough representatives of each confusion matrix quadrant for each class, one could consider formulating a microF1. The case where the number of classes is small, the dataset is sizeable and enough memory is available would be favorable to that end.

$\mathcal{L}_{\overline{\mathit{F1}}}$ will be used to benchmark against our proposed \emph{sigmoidF1} loss.


Thanks to the definitions of smooth confusion matrix metrics above, we can now write $\mathcal{L}_{\widetilde{\mathit{F1}}}$.

\begin{equation}\label{eq:sigmoidF1}
\mathcal{L}_{\widetilde{\mathit{F1}}}= \frac{\widetilde{\mathit{tp}}}{2 \widetilde{\mathit{tp}}+ \widetilde{\mathit{fn}}+ \widetilde{\mathit{fp}}}
\end{equation}

\begin{figure}[htbp]
\centering
\includegraphics[width=.9\linewidth]{./images/sigmoid.pdf}
\caption{\label{fig:sigmoid}
insert caption here}
\end{figure}

A similar method was proposed outside of the context of neural networks: the \emph{Maximum F1-score criterion} for automatic mispronunciation detection as an objective function to a Gaussian Mixture Model-hidden Markov model (GMM-HMM)\cite{sigmoid}.

Similarly to the focal loss \cite{focalLoss}, sigmoidF1 loss deals with class imbalance and robustness to outliers.

\todo{statistical robustness assessment}


% Given the presence of the step indicator function \(\sum \mathds{1}_{\mathbf{p_i} \geq b}\), \(F_\beta\) is not differentiable for gradient based methods. One way of surpassing that problem is to use a surrogate.

% \subsection{soft F1 score}
% \label{sec:org3ca83ef}

 % with smooth confusion matrix entries :



% /softF1/ is
% $$\mathcal{L}_{\text {Pred}}=\sum_{i, j}\left(\mathbf{y}_{i j}-\hat{\mathbf{y}}_{i j}\right)^{2}$$

% \subsection{sigmoidF1 score}
% \label{sec:orgc5d29d7}


% \begin{figure}[htbp]
% \centering
% \includegraphics[width=.9\linewidth]{./images/sigmoid.pdf}
% \caption{\label{fig:sigmoid}
% Sigmoid function with different values for $\beta$ (steepness) \& $\eta$ (offset)}
% \end{figure}

%  with \(S(u)\), the confusion matrix entries then become

% \begin{equation}\label{eq:sigmoidF1}
% \widetilde{\mathit{tp}}=\sum S(\hat{\mathbf{y}}) \odot \mathbf{y} \quad\widetilde{\mathit{fp}}= \sum S(\hat{\mathbf{y}}) \odot (\mathbf{1} - \mathbf{y}) \quad \widetilde{\mathit{fn}}= \sum (\mathbf{1} - S(\hat{\mathbf{y}})) \odot \mathbf{y}
% \end{equation}

% And thus

% \begin{equation}
% \mathcal{L}_{\text {sigmoidF1}}= \frac{\widetilde{\mathit{tp}}}{2 \widetilde{\mathit{tp}}+ \widetilde{\mathit{fn}}+ \widetilde{\mathit{fp}}}
% \end{equation}

% \doubt{mention smooth hinge loss} \cite{smoothHinge}

% $\doublewidetilde{tp}$
% https://tex.stackexchange.com/questions/321231/double-widetilde
% doesn't work


% \todo{explain batch size mathematically for F1 surrogate losses}


%%% Local Variables:
%%% mode: latex
%%% TeX-master: "../main"
%%% End:

% !TEX root = ../main.tex

\section{Experimental Setup}
\label{sec:orgb44ba25}

We test multilabel learning using our proposed sigmoidF1 loss function on four datasets across different modalities, image and text. 
The datasets fit the FIMPUL characteristics. In particular, they are best modeled with full-instance learning as the entire image or the full text is predictive of the instance's labels. The next section and Table~\ref{table:datasets} describe our four datasets. The learning architecture (classification head attached to a pretrained net) is then described before addressing hyperparameter tuning and reproducibility remarks.

\subsection{Datasets}

Our first dataset comes from the vision domain and consists of movie posters and their genres (e.g., \emph{action}, \emph{comedy}).\footnote{Labels available at \url{https://tinyurl.com/y7ydyedu} and prescraped images from IMDB at \url{https://tinyurl.com/y7lfpvlx}.} The posters and labels have been extracted from IMDB and the dataset was previously used for per-class, post-training thresholding \citep{moviePosters} (see Section~\ref{sec:org2aceb9f}). The genre labels in this dataset are not mutually exclusive and of varying counts per movie. 

We use the newly created \emph{arXiv dataset}\footnote{Available at \url{https://www.kaggle.com/Cornell-University/arxiv}} with over 1.7 million open source articles and their metadata. Our experiments use the abstracts and categories that are suitably non-mutually exclusive and of varying counts per example. There is a longer history of using arXiv to create research datasets; the dataset we use is not to be confused with an earlier long document dataset that only features 11 classes~\citep{oldarXiv}, but was used in a recent long transformer publication~\cite{bigBird}. The limited number of labeled classes render the older dataset unsuitable for our experiments.  We write \textit{arXiv2020} for the subset of the \emph{arXiv dataset} that only contains documents published in 2020, given limited computing power. This results in around 26k documents.

To the best of our knowledge, ML-NET~\cite{multitaskLabel} is the state-of-the-art among \emph{fit-algorithm-to-data} methods for multilabel learning with unknown label count on text (the work does not differentiate full-instance and multi-instance learning, see Section~\ref{sec:org2aceb9f}). Among the three datasets used for benchmarking ML-NET, the cancerHallmark~\citep{cancerHallmarks}\footnote{Available at \url{https://github.com/sb895/Hallmarks-of-Cancer}} and chemicalExposure~\citep{chemExposure}\footnote{Available at \url{https://github.com/sb895/chemical-exposure-information-corpus}} datasets are of multi-instance multilabel nature: several expressions are annotated within each paper abstracts. The third dataset diagnosisCodes could not be obtained (neither from the authors of ML-NET nor of the original paper~\cite{diagnosisCode}). We treat cancerHallmarks and chemicalExposure as full-instance datasets by aggregating expression labels over each example, as was done for ML-NET.

\begin{table}
\caption{Descriptive statistics of our experimental datasets.}
\label{table:datasets}
\centering
% \begin{adjustbox}{max width=\textwidth}
\begin{tabular}{l rrrr}
\toprule
& & & Average & Number of \\
& Type & Classes & label count & examples \\
\midrule
moviePosters & image & 28 & 2.165 & 37,632\\
arXiv2020 & text & 155 & 1.888 & 26,558\\ 
chemExposure & text & 38 & 6.116 & 3,661\\
cancerHallmarks\hspace{-.7em}  & text & 33 & 3.501 & 1,582\\
\bottomrule
\end{tabular}
% \end{adjustbox}
\end{table}

\subsection{Learning Framework}

The learning framework consists of two parts: a pretrained deep neural network and a classification head (see Figure \ref{fig:architecture}); the classification head is where we slot in alternative loss functions.

\begin{figure*}[htbp]
\centering
\includegraphics[width=.9\linewidth]{./images/architecture.png}
\caption{\label{fig:architecture}
Pretrained Network and Classification Head for sigmoidF1 \todo{some tweaks}}
\end{figure*}

Since the focus of this paper is in comparing loss functions and not neural network architectures, we chose efficient network architectures in terms of accuracy and computation.
For the moviePoster image dataset, we use a MobileNetV2~\cite{mobileNet} architecture that was pretrained on ImageNet~\cite{imagenet}. This network architecture is typically used for inference on small computing devices (e.g., smartphones). We use a version of MobileNetV2 already stripped off of its original classification head.\footnote{The pretrained network can be loaded here \url{https://tfhub.dev/google/imagenet/mobilenet_v2_100_224/feature_vector/4}.}
For the three text datasets, we use DistilBert~\cite{distilBert} as implemented in Hugging Face. This is a particularly efficient instance of the BERT model.\footnote{The pretrained network can be loaded here \url{https://huggingface.co/transformers/model_doc/distilbert.html}.}.
In both cases, we use the final pretrained layer as an embedding of the input. In order to make sure that the results of different loss functions are comparable, we fix the model weights of the pretrained MobileNetV2 and DistilBert and keep the hyperparameter values that were used to train them from scratch. We also initialize the model weights with the same TensorFlow internal random seeds across training sessions.

A similar classification head is used for both MobileNetV2 and DistilBert. It consists of a latent representation layer (the final pretrained layer mentioned above) connected with a RELU activation. This layer is linked to a final classification layer with a linear activation. The dimension of the final layer is equal to the number of classes in the dataset. When computing focalLoss and crossEntropy, a softmax transformation transforms the unbounded last layer to a $[0,1]$ bounded vector. When computing sigmoidF1 loss, a sigmoid transformation is operated instead, which results in a sparser $[0,1]$ vector. At inference time, the last layer is used for prediction and is bounded with a softmax function. A threshold must then be chosen at evaluation time to compute different metrics. Figure \ref{fig:architecture} depicts this learning framework. Thresholding regimes at inference time are further discussed in Sections~\ref{sec:evalMetrics} and~\ref{subsec:thresh}.

\subsection{Losses Compared}
\label{section:losssescompared}
\mdr{Briefly state/summarize which losses we will be comparing and why those.}
 
\subsection{Evaluation Metrics}
\label{sec:evalMetrics}

In our experimental evaluation, we consider a suite of metrics that are commonly used in the evaluation of multilabel classification to measure the effectiveness of multilabel prediction. Such metrics are based on the confusion matrix tjat we detailed in Section~\ref{section:method} and for which we provided smoothed surrogates to optimize directly.

When true positives and false positives are used, recall that \(t p=\sum_{i \in Y^{+}} \mathds{1}_{\mathbf{p_i} \geq t}\) and \(f p=\sum_{i \in Y^{-}} \mathds{1}_{\mathbf{p_i} \geq t}\), and thus a threshold \(t\) must be set. We set \(t = 0.5\), as is commonly done \todo{add source}.
For the two medical datasets, cancerHallmarks and chemicalExposure, information is a lot more sparse, we thus set the evaluation metrics threshold at 0.05 and train for 500 epochs until reaching convergence. 

Extending \(F_1\) to multi-class binary classification amounts to deciding wether or not to pool classes.
In a first pooled iteration, macro \(F_1\)~\cite{multilabelMetrics} equates to creating a single 2x2 confusion matrix for all classes:
\begin{equation}
F_1^{macro} = \frac{\sum tp_c}{2 \sum tp_c + \sum fn_c + \sum fp_c} \quad for \quad c \in C.
\end{equation}
Micro \(F_1\) \cite{threshForF1, multilabelMetrics} amounts to creating one confusion matrix per class or unpooling:
\begin{equation}
F_1^{micro} = \frac{1}{c} \sum_{j=1}^c F_1.
\end{equation}
Weighted micro \(F_1\)~\cite{weightedMetrics} is similar but includes weighing to account for class imbalance, i.e., weighing each class by the number of ground-truth positives:
\begin{equation}
F_1^{weighted} = \frac{1}{c} \sum_{j=1}^c n_j F_1 \text{ where } n_j = \sum_i \mathds{1}_{\mathbf{y_i^j} = 1}.
\end{equation}
%
We also define precision and recall
%
\begin{equation}
\begin{aligned} P &=\frac{t p}{t p+f p} \\ R &=\frac{t p}{t p+f n}=\frac{t p}{\left|Y^{+}\right|} \end{aligned}
\end{equation}
%
\gab{check if precision and recall is at micro or macro level}
In our experiments, we report on weightedF1, microF1, macroF1, Precision and Recall. Macro scores do not differentiate multiclass unilabel classification from multiclass multilabel classification. Micro scores treat classes separately. The \emph{weighted} micro F1 score is a further refinement where class scores are weighted by their representation in the dataset. We thus focus our discussion of experimental results around weightedF1 as we consider this to be the most representative for success on FIMPUL problems. 

There is an interaction between our optimization on sigmoidF1 and our evaluation using (weighted) F1 metrics. If our approach of optimizing for an F1 surrogate succeeds, we expect higher values on F1-related metrics during evaluation. For this reason, we consider and discuss not a single, but multiple metrics.

\subsection{Hyperparameters and Reproducibility}

We choose to ignore classes that are underrepresented, in order to give the model a fair chance at learning from at least a few examples. We define underrepresentation as a global irrelevance threshold $b$ for classes: any class $c$ that is represented less than $b$ times is considered irrelevant. We decided to set an irrelevance threshold $b$ on all datasets prior to conducting experiments, so as to not finetune for that feature. It was set to 1000 for both \emph{arXiv2020} (145 of the original 155 classes remaining) and \emph{moviePosters} (14 of the 28 classes remaining) and at 10 for \emph{chemicalExposure} (\todo{XX} of the 38 classes remaining) and \emph{cancerHallmarks} (\todo{XX} of the 33 classes remaining), in proportion to the number of classes and labels in each dataset. The sensitivity study below illustrates what happens when we let $t$ vary over the arXiv2020 dataset.

Batch size is set at a high value of 256. We thus increase accuracy over traditional losses~\cite{bigBS}, but also allow heterogeneity in the examples within the batch, thus collecting enough values in each quadrant of the confusion matrix. In a future research, it would be interesting to establish the minimum required batch size for sigmoidF1. \daan{I think we should discuss this need for a large batch size in a bit more detail in 3 and then we can reference that here and say we set it sufficiently high. We really should have an experiment with this parameter, but there is no time to still run that.} \gab{I do have few clean experiments with bs: 128}

Regarding the sigmoidF1 hyperparameters $\beta$ and $\eta$, we performed a grid search with the values in the range $[1, 6, 7, 8, 9, 10, 20,$ $30]$ for $\beta$ and $[0, 0.5, 1, 2]$ for $\eta$.
In our experiments, we evaluate the sensitivity of our method to these hyperparameters (see Figure~\ref{fig:sigmoid}).

We made sure to split the data in the same training, validation and test sets for each loss function. Our code, dataset splits and other settings are shared to ensure reproducibility of our results. 

We performed our experiments on Amazon Web Services cloud machines with parallelization on up to 16 GPUs \textit{p2.16xlarge}, with TensorFlow 2 as a gradient-descent backend.

% !TEX root = ../main.tex

\section{Experimental Results}
\label{sec:orgc23a664}

We show results on four different datasets, moviePosters, cancerHallmarks, chemicalExposure and Arxiv2020.
Interestingly, tencentLoss and focalLoss, which are specifically tailored for sparse data, do not perform well here. Note that macroSoftF1 can be particularly performent without requiring hyperparameter tuning. Across Tables \ref{tab:chemicalExposure, tab:moviePosters, tab:arxiv2020, tab:cancerHallmarks} we see that the smooth sigmoidF1 outperforms other existing losses. Although only the metrics@0.5 are shown, where 0.5 is the dichotomization threshold, the ablation study and our open sourced code further illustrates the consistent good performance of sigmoidF1 versus other losses.


For the two medical datasets cancerHallmarks and chemicalExposure, information is a lot more sparse, we thus set the evaluation metrics threshold at 0.05 and train for 500 epochs until reaching convergence. 

\subsection{Ablation Study - Arxiv2020}

We performed an analysis of sensitivity to different amount of labels. For this, the irrelevance threshold (see definition in Section \ref{sec:orgb44ba25}) $t$ was set to the values 0, 10, 100 and 1000. This time, results are shown with dichotomization thresholds of 0.1 to 0.9. \todo{see if keep it}

\todo{sensitivity to hyperparametertuning plot}

\begin{figure}[htbp]
\centering
\includegraphics[width=.9\linewidth]{./images/ablation.pdf}
\caption{\label{fig:ablation}
Sigmoid function with different values for $\beta$ (steepness) \& $\eta$ (offset)}
\end{figure}


\todo{change to percentage}

\begin{table}
  \caption{Movie posters (CNN). \mdr{Explain what we see.}}
  \label{tab:moviePosters}
\centering
\begin{tabular}{l ccccc}
\toprule 
Loss  & \rotatebox{90}{macroF1 @ 0.5} & \rotatebox{90}{microF1 @ 0.5} & \rotatebox{90}{weightedF1 @ 0.5} & \rotatebox{90}{Precision @ 0.5} & \rotatebox{90}{Recall @ 0.5}\\ 
\midrule
$\mathcal{L}_{\text {CE}}$ & 0.051 & 0.186 & 0.149 & 0.090 & 0.042 \\ 
$\mathcal{L}_{\text {FL}}$ & 0.055 & 0.192 & 0.154 & 0.115 & – \\
% $\mathcal{L}_{\text {CE+N}}$ & 0 & 0 & 0 & 0 & 0 \\
$\mathcal{L}_{\text {macroSoftF1}}$ & 0.136 & 0.207 & 0.243 & \textbf{0.105} & 0.190 \\
$\mathcal{L}_{\text {sigmoidF1}}$ & \textbf{0.158} & \textbf{0.224} & \textbf{0.300} & 0.104 & \textbf{0.557} \\ % run aa424792a57e4208ad1805cd6e63f8e6
\bottomrule
\end{tabular}
\end{table}


\begin{table}
  \caption{Arxiv (distillBERT 2020), frozen pretrained weights 100 epochs, min-label-thresh: 1000}
  \label{tab:arxiv2020}  
\centering
\begin{tabular}{l ccccc}
\toprule
Loss  & \rotatebox{90}{macroF @ 0.5} & \rotatebox{90}{microF1 @ 0.5} & \rotatebox{90}{weightedF1 @ 0.5} & \rotatebox{90}{Precision @ 0.5} & \rotatebox{90}{Recall @ 0.5}\\ 
\midrule
$\mathcal{L}_{\text {CE}}$ & 0.093 & 0.106 & 0.106 & 0.096 & – \\ % Run 71ef078f975649d5b3d897e504bc638b
$\mathcal{L}_{\text {FL}}$ & 0.008 & 0.011 & 0.009 & 0.054 & 0.954 \\
% $\mathcal{L}_{\text {CE+N}}$ & 0 & 0 & 0 & 0 & 0 \\
$\mathcal{L}_{\text {macroSoftF1}}$ & 0.077 & 0.088 & 0.087 & 0.100 & – \\ % run 405a6e6851e84a89a82313251a7a36e8 (18-19 jan)
$\mathcal{L}_{\text {sigmoidF1}}$ & \textbf{0.093} & \textbf{0.106} & \textbf{0.106} & \textbf{0.096} & \textbf{–} \\ % run bd478ca55eb64cc78d9ad0f25accce35 (18-19 jan)
\bottomrule
\end{tabular}
\end{table}

\begin{table}
  \caption{Cancer Hallmarks}
  \label{tab:cancerHallmarks}
\centering
\begin{tabular}{l ccccc}
\toprule 
Loss  & \rotatebox{90}{macroF1 @ 0.05} & \rotatebox{90}{microF1 @ 0.05} & \rotatebox{90}{weightedF1 @ 0.05} & \rotatebox{90}{Precision @ 0.05} & \rotatebox{90}{Recall @ 0.05}\\ 
\midrule
$\mathcal{L}_{\text {CE}}$ & 0.0 & 0.0 & 0.0 & 0.0 & 0.0 \\ 
$\mathcal{L}_{\text {FL}}$ & 0.044 & 0.190 & 0.108 & 0.071 & 0.055 \\
% $\mathcal{L}_{\text {CE+N}}$ & 0 & 0 & 0 & 0 & 0 \\
$\mathcal{L}_{\text {macroSoftF1}}$ & 0.098 & 0.176 & 0.170 & 0.089 & 0.131 \\
$\mathcal{L}_{\text {sigmoidF1}}$ & \textbf{0.095} & \textbf{0.313} & \textbf{0.202} & \textbf{0.059} & \textbf{0.264} \\ % run e145056949424b02bfc83cc57af38374
\bottomrule
\end{tabular}
\end{table}


\begin{table}
  \caption{Chemical Exposure}
  \label{tab:chemicalExposure}
\centering
\begin{tabular}{l ccccc}
\toprule
Loss  & \rotatebox{90}{macroF1 @ 0.05} & \rotatebox{90}{microF1 @ 0.05} & \rotatebox{90}{weightedF1 @ 0.05} & \rotatebox{90}{Precision @ 0.05} & \rotatebox{90}{Recall @ 0.05}\\ 
\midrule
$\mathcal{L}_{\text {CE}}$ & 0.012 & 0.058 & 0.051 & 0.047 & 0.007 \\ % Run 71ef078f975649d5b3d897e504bc638b
$\mathcal{L}_{\text {FL}}$ & 0.0934 & 0.348 & 0.268 & 0.130 & 0.091 \\
$\mathcal{L}_{\text {macroSoftF1}}$ & 0.133 & 0.194 & 0.218 & 0.155 & 0.138 \\ % run 405a6e6851e84a89a82313251a7a36e8 (18-19 jan)
$\mathcal{L}_{\text {sigmoidF1}}$ & \textbf{0.113} & \textbf{0.432} & \textbf{0.319} & \textbf{0.091} & \textbf{0.188} \\ % run a30825efe9c94a24bc46a9c71a5f8646 E: 0.5 S: 0.6
\bottomrule
\end{tabular}
\end{table}


%%% Local Variables:
%%% mode: latex
%%% TeX-master: "../main"
%%% End:

% !TEX root = ../main.tex

\section{Related Work}
\label{sec:org2aceb9f}

% Beyond identifying object types (see YOLO~\cite{YOLO} and its successors),
% performing face recognition (see FaceNet\cite{FaceNet} and its successors) on
% segments of an image, n
% Neural networks are increasingly becoming better at predicting more abstract
% concepts via deeper networks, representation learning and
% self-supervision~\citep[see, e.g.,][]{SS,Rep}. There is a significant volume
% of recent work on building neural networks with a high-level of abstract
% understanding in the embedding space~\mdr{REF}. However, research on
% developing optimization frameworks that are adapted for these abstract
% concepts in the output space is limited. In the next sections we detail
% some of this related work and discuss it relation to the research in this paper.

Existing algorithmic solutions to deal with multilabel tasks can be divided into \emph{fit-data-to-algorithm} solutions, which map FIMPUL problems to a known problem formulation like multiclass uni-label classification, and \emph{fit-algorithm-to-data} solutions, which adapt existing classification
algorithms to the problem at hand~\citep{multilabelMethods}.

\subsection{Fit-data-to-algorithm}
Commonly, in fit-data-to-algorithm solutions, cross-entropy losses are used at training time and thresholding is done at inference time to determine how many labels should be assigned to an instance. This has also been called multilabel reduction ~\citep{multilabelReduction}, to be distinguished from multiclass to binary classifications ~\citep{multiclassToBinary1, multiclassToBinary2, multiclassToBinary3}.

We can further make the distinction between One-versus-all (OVA) and Pick-all-labels (PAL) solutions ~\citep{multilabelReduction} (see also~\ref{section:background}).

In OVA (a.k.a binary relevance model), one reduces the classification problem to independent binary classifications~\citep{OVA1, hammingLoss, OVA2}. OVA Theory \cite{OVATheory}.

In PAL , one ... \citep{labelPowerset, extremeClassification, PAL}. The \textit{label powerset} approach considers each unique set of labels as one class in the transformed setting~\cite{labelPowerset}.

In Pick-One-Label (POL), a single multiclass example is created by randomly sampling a positive label \cite{PAL, extremeClassification}.

Note that Menon et. al. propose a normalized versions of OVA and PAL~\cite{multilabelReduction}.

% (e.g., an instance labeled \textit{Thriller} and \textit{Action}, results in the creation of the class \textit{Thriller and action}).
Alternatively, \textit{ranking by pairwise comparison} is a solution where the dataset is duplicated for each possible label pairs. Each duplicated dataset has therefore two classes and only contains instances that have at least one of the labels in the label pair. Different ranking methods exist~\cite{pairwiseBinary, pairwiseNet}.

More recently, hierarchical datasets such as DBpedia~\citep{lehmann2015dbpedia} are used to finetune BERT-based models~\cite{XLNet, bigBird};  the latter publications use cross-entropy to predict the labels.

\subsection{Fit-algorithm-to-data}
In the fit-algorithm-to-data solutions, elements of the learning
algorithm are changed (such as the back propagation procedure or the task).
Early representatives of fit-algorithm-to-data stem from heterogenous domains
of machine learning. MultiLabel k-Nearest Neighbors \cite{ML-KNN},
MultiLabel Decision Tree \cite{ML-DT}, Ranking Support Vector Machine
\cite{multilabelSVM} and Backpropagation for MultiLabel Learning
\cite{multilabelBackprop}. More recently, two papers introduced the idea of
multi-task learning for \emph{label prediction} and \emph{label count
prediction} for text data \cite[ML\(_{\text{NET}}\)][]{multitaskLabel} and image
data \cite{multitaskLabelImages, tencent}. The latter research is loosely
catered towards object detection and is thus out-of-scope: local elements in a picture are predicted that tend to be uni-label as defined by the ground-truth (e.g., cat, flower, vase, person, bottle
etc.).

An important limitation shared by both \emph{fit-data-to-algorithm} and \emph{fit-algorithm-to-data} is the lack of a holistic approach for both label count and label prediction.

\begin{figure*}[t!]
\centering
\includegraphics[width=.9\linewidth]{./images/betaEtaResized.png}
\vspace{1\baselineskip}
\caption{\label{fig:betaEta}
DistilBert (NLP) + classification head on arXiv2020 – different scores at a 0.5 threshold for different values of $\eta$ and $\beta$ in a sampling region similar to Figure~\ref{fig:sigmoid}}
\end{figure*}

\subsection{Thresholding}
\label{subsec:thresh}

Machine learning prediction tasks' output are probabilistic (or a reversible transformation of a probabilistic measure such as a sigmoid or a softmax function).
At training time, these outputs are compared to binary
values in the case of binary encoding of classes.

At inference time, if the number $n_i$ of labels to be predicted per example is known a priori, it is natural to assign the $top_{n_i}$ predictions to that example~\cite{lossTopKError, topKmulticlassSVM}.
If the number of labels per example is unknown a priori,  at inference time the question remains as to how to extract information about the number of labels to assign to each example, aside from the propensity of labels to be assigned.
This is generally done via a \emph{decision threshold}, that can be set globally for all examples, and be optimized for specificity or sensitivity~\cite{decisionThreshold}.

Thresholding across classes or examples can be an issue when the number of labels to predict is unknown. Some variants of cross-entropy loss accommodate imbalanced label data~\cite{focalLoss}, but remain agnostic to the number of labels to predict.
Solutions include determining an ideal global \emph{threshold} depending on the use-case at hand~\cite{threshForF1}, or per-class-thresholding after training~\cite{moviePosters} and eventually abstracting the threshold away via a \emph{soft-F1} measure~\cite{softF1}.
%In the latter two cases, the task is to predict genre from movie posters.

In our method, this threshold is defined implicitly, thanks to the use of metrics that already penalize for
wrong label counts.


\subsection{Metrics as losses}

In a number of learning to rank tasks~\cite{LTR}, a model's out of sample accuracy is measured on metrics such as AUROC, F1 score, etc. These reflect an objective catered towards evaluating the model over an entire ranking. Due to the lack of differentiability, these metrics cannot be directly used as loss
functions at training time (in-sample). \citet{optimizableLosses} propose a general framework for deriving
decomposable surrogates from some of these metrics.
Recently, a similar work has been proposed to train a Convolutional
Neural Network (CNN) from scratch with a few millions of images and hundreds
of labels specifically for multilabel tasks \cite{tencent}; this task is loosely related to object detection, similar to \cite{multitaskLabelImages}.
In our research, we instead propose decomposable surrogates of the classical confusion matrix metrics and in particular \emph{sigmoidF1}, tailored to the problem at hand.

% The proposed method is positioned in the lineage of \emph{algorithm
% adaptation}, using \emph{metric as losses} and allowing for dynamic
% \emph{thresholding}.

% This section will be guided by the previous section's formulation of the multitags problem, we will therefore focus on \emph{fit-algorithm-to-data}, \emph{metrics as losses} and \emph{thresholding}.

% \subsection{fit-algorithm-to-data}
% \label{sec:org150a474}

% % Early representatives of \emph{fit-algorithm-to-data} stem from heterogenous domains of machine learning. Multi-Label k-Nearest Neighbors \cite{ML-KNN}, Multi-Label Decision Tree \cite{ML-DT}, Ranking Support Vector Machine \cite{multilabelSVM} and Backpropagation for Multi-Label Learning \cite{multilabelBackprop}. More recently, two papers introduced the idea of multitask learning for \emph{label prediction} and \emph{label count prediction} for text (ML\(_{\text{NET}}\)) \cite{multitaskLabel} and image \cite{multitaskLabelImages} data. The latter research is loosely catered towards object detection (although not formally presented as such) and is thus out-of-scope: elements in a picture are predicted that tend to be unilabel as defined by the groundtruth (e.g. cat, flower, vase, person, bottle etc.).

% \subsection{Metrics as losses}
% \label{sec:orgb0a9d21}

% Often, machine learning post-training evaluation metrics (e.g. AUROC, F1) are not differentiable. There are motivations \todo{which motivations} for optimizing a model directly on a metric at training time. A general framework for AUC, AUROC and F1 is presented in \cite{optimizableLosses}, but the proposed F1 surrogate remains short of being explicitly derived for stochastic gradient descent. \todo{check again with the authors if I can't get inspired from their work}. Recently, a similar work has been proposed to train a Convolutional Neural Network (CNN) from scratch with a few millions of images and hundreds of labels specifically for multilabel tasks \cite{tencent}. This task is loosely related to object detection, similarly to \cite{multitaskLabelImages} mentioned in the previous paragraph.


% in reformulating loss functions to accomodate sparsity in the data, to optimize directly for the metric at hand or to do thresholding posthoc (see movie posters).

% \subsection{Thresholding}
% \label{sec:org8295f09}

% \emph{thresholding} accross classes or examples can be an issue as soon as the number of labels to predict is unknown. Certain variants of cross-entropy loss accommodate imbalanced label data  \cite{focalLoss}, but remain agnostic towards the number of labels to predict. Solutions have been tailored to that end, starting with determining an ideal global \emph{threshold} depending on use-cases \cite{threshForF1}, or per-class-thresholding after training \cite{moviePosters} and eventually abstracting the threshold away via a \emph{soft-F1} measure \cite{softF1} \todo{say more about this method}. In the latter two cases, the task is to predict genre from movie posters.



% The proposed method is positioned in the lineage of \emph{fit-algorithm-to-data}, using \emph{metric as losses} and allowing for dynamic \emph{thresholding}.

% \todo{compare to this:} \cite{lossComp}

% \todo{Hamming Loss}
% % \todo{Precision@K, Recall@K, NDCG@K}
% \todo{MLTSVM loss and the three-way loss inspired by it} \cite{MLTSVM} and \cite{MLTSVMThreeway}

% We propose a dynamic thresholding mechanism auto-tuned at training time.


% ** weak labels
% (unsure the labels are correct)

% - https://people.cs.pitt.edu/~kovashka/ye_zhang_kovashka_iccv2019_cap2det.pdf


% ** implementations

% *** movies

%  [[https://www.analyticsvidhya.com/blog/2019/04/build-first-multi-label-image-classification-model-python/][movie posters with classes]].

%  They have movie titles in them

% *** pretrained resnet on multilabel

%  https://github.com/Tencent/tencent-ml-images

% What happens when using a Resnet pretrained on multilabels

% *** soft F1 score loss

%  https://github.com/ashrefm/multi-label-soft-f1

% https://www.analyticsvidhya.com/blog/2019/04/build-first-multi-label-image-classification-model-python/



% /Optimizing directly for macro F1: By introducing the macro soft-F1 loss, we could train the model to directly increase the metric we care about: the macro F1-score @ threshold 0.5. We could clearly observe the alignment during training and evaluation on successive epochs. When using this loss, we do not have to tune the decision threshold any more. Imagine a multi-label classification system with hundreds of labels, how unstable the system will be if we have to continuously update the optimal threshold for each label. The macro soft-F1 loss comes to the rescue. By using it, we can keep all thresholds fixed at 0.5 and still get an optimal performance from the training process./



%%% Local Variables:
%%% mode: latex
%%% TeX-master: "../main"
%%% End:

% !TEX root = ../main.tex

\section{Conclusions}
\label{sec:orged3d8a1}
% \mdr{Structure the conclusion in five paragraphs, devoted to the following questions:}
% \begin{itemize}[leftmargin=*]
% \item What did we do
% \item What did we find
% \item What are the implications
% \item What are the limitations
% \item What should we do next
% \end{itemize}


\paragraph{Results}
In this paper, we framed a new problem that we call Full-Instance, Multilabel Prediction for Unknown Label count (FIMPUL). FIMPUL is a specific type of multiclass classification problem in which 1) labels are not mutually exclusive, 2) instances need to be consumed in its entirety to be labelled,
and 3) the number of labels to assign at inference time unknown.
% Advances in deep learning research facilitate more abstract representations of the input space for multiple modalities, these can be used to effectively solve downstream tasks (e.g., with models of the BERT and CNN model families) that relate to FIMPUL problems.
To solve multilabel learning tasks, existing optimization frameworks are typically based on variations of the cross-entropy loss.

To address FIMPUL problems, we propose a general loss framework for confusion matrix metrics (CoMMaL), and perform experiments with a specific loss function from the CoMMaL framework: the \emph{sigmoidF1} loss. We find that the \emph{sigmoidF1} loss can achieve significantly better results for most metrics on four diverse datasets and that the \emph{sigmoidF1} loss outperforms other losses on the weightedF1 metric.
More generally, our smooth formulation of confusion matrix metrics allows us to optimize directly for these metrics that are usually reserved for the evaluation phase.

\paragraph{Limitations}
We evaluate the CoMMaL framework and \emph{sigmoidF1} loss function only on FIMPUL problems. It may be worth considering the effectiveness of the CoMMaL framework in different domains. More experimentation is also needed to find a proper heuristic for finetuning the hyperparameters of the \emph{sigmoidF1} loss. The proposed \emph{unboundedF1} counterpart does not require tuning and delivered better results than existing multiclass losses on most metrics; it can act as a mathematically less robust approximation of \emph{sigmoidF1}.

\paragraph{Future work}
In future work, we want to test the limits of CoMMaL and the \emph{sigmoidF1} loss both within and beyond the FIMPUL setting.
%But first, if provided enough computing power, one could perform the same experiments with micro losses: given enough representatives of each confusion matrix quadrant for each class, one could consider formulating a micro F1 as a loss.
%The case where the number of classes is small, the dataset is sizeable and enough memory is available (especially with the recent advances in model parallelism) would be favorable to that end.
%
%% future work %%
% datasets
% labeling setting
% neural architecture
% SOTA transfer learning
% train from scratch


\emph{multi-instance} learning~\citep[e.g.,][]{multiInstance,multiInstanceMultiLabel}, for which it is considered natural to first segment an image, text or sound before performing prediction on each of the segments

A first step within the FIMPUL setting, could be to use more robust transfer learning / finetuning procedures, for example with dynamic weight freezing for finetuning~\cite{ULMFit}. Alternatively, we would like to implement the smooth losses to train a CNN or a BERT model for FIMPUL tasks from scratch (c.f., \cite{tencent} and \cite{focalLoss}). If training from scratch, it might then be interesting to combine the proposed loss functions with representation learning \cite{unsupervisedImage,highResRepresentation} or self-supervised learning, in order to model abstract relationships between the labels.

Furthermore, we propose to tackle other multilabel settings, such as hierarchical multilabel classification \cite{HARAM}, active learning \cite{activeLearningMultiLabel}, or extreme multilabel prediction \cite{extremeMultilabelText, extremeSIGIR, extremeClassification}, where the number of classes ranges in the tens of thousands. \gab{move this sentence to the background?} More generally, the \emph{sigmoidF1} loss could be tested on any model that uses F1 score as an evaluation metric such as AC-SUM-GAN \cite{AC-SUM-GAN}.
Recently emerging \textit{holistic} content labeling tasks might be another promising testing ground for CoMMaL and \emph{sigmoidF1}. Holistic labeling refers to labels given to an entire content (full-instance) at different levels of abstraction. A dataset was recently released for \emph{Large Scale Holistic Video Understanding}~\cite{holisticVideoData}.\footnote{Available at \url{https://github.com/holistic-video-understanding/HVU-Dataset}}
\if0
\vspace{\baselineskip}

In the mean time, if this paper is accepted, we will release our results on Kaggle website for the arXiv multilabel text classification task\footnote{\url{https://www.kaggle.com/Cornell-University/arxiv/tasks?taskId=1757}} in the months to come.
\fi



%%% Local Variables:
%%% mode: latex
%%% TeX-master: "../main"
%%% End:


\section*{Reproducibility}
% To facilitate the reproducibility of the reported results, t
This work only made use of publicly available data and our experimental implementation is publicly available at ...

\begin{acks}
 This work was supported by many people.
 All content represents the opinion of the authors, which is not necessarily shared or endorsed by their respective employers and/or sponsors.
\end{acks}

\bibliographystyle{ACM-Reference-Format}
\bibliography{references}

\end{document}


%%% Local Variables:
%%% mode: latex
%%% TeX-master: t
%%% End:
