% Created 2020-12-27 Sun 23:04
% Intended LaTeX compiler: pdflatex
\documentclass[sigconf,natbib,screen=true,review=true,anonymous]{acmart}
\usepackage[utf8]{inputenc}
\usepackage[T1]{fontenc}
\usepackage{graphicx}
\usepackage{grffile}
\usepackage{longtable}
\usepackage{wrapfig}
\usepackage{rotating}
\usepackage[normalem]{ulem}
\usepackage{amsmath}
\usepackage{textcomp}
\usepackage{amssymb}
\usepackage{capt-of}
\usepackage{hyperref}
\usepackage{minted}
% We'll get the submission number fro the submission system
\acmSubmissionID{xx}
% !TEX root = ./main.tex

% list your packages
\usepackage[utf8]{inputenc}
\usepackage[T1]{fontenc}
\usepackage{graphicx}
\usepackage{grffile}
\usepackage{longtable}
\usepackage{wrapfig}
\usepackage{rotating}
\usepackage[normalem]{ulem}
%\usepackage{amsmath}
\usepackage{textcomp}
%\usepackage{amssymb}
\usepackage{capt-of}
\usepackage{hyperref}
%\usepackage{minted}

\usepackage{algorithmic}
\usepackage{graphicx}
\usepackage{xcolor}
\usepackage{multirow} 
\usepackage{xcolor}
\usepackage[ruled,linesnumbered,boxed]{algorithm2e}
\usepackage[inline]{enumitem}
\usepackage{booktabs}
\usepackage[justification=centering, skip=0pt]{caption}
\usepackage{acronym}
\usepackage{url}
% \usepackage{graphicx}
% \usepackage{adjustbox}

\usepackage{xcolor}

%\usepackage{amsmath}
\usepackage{dsfont}
\usepackage{color}

\usepackage{subcaption}
% !TEX root = ./main.tex

% list your definitions
\newcommand{\mdr}[1]{\textcolor{orange}{\textbf{[#1]}}}

\newcommand\todo[1]{\textcolor{red}{TODO : #1}}
\newcommand\doubt[1]{\textcolor{orange}{DOUBT : #1}}
% \newcommand\todo[1]{} % uncomment to hide comments
% \newcommand\doubt[1]{} % uncomment to hide comments

\hypersetup{
 pdfauthor={Gabriel Bénédict},
 pdftitle={},
 pdfkeywords={},
 pdfsubject={},
 pdfcreator={Emacs 26.1 (Org mode 9.1.14)}, 
 pdflang={English}}

% !TEX root = ./main.tex

\author{Gabriel B\'en\'edict}
\affiliation{
\institution{RTL \& University of Amsterdam} 
\city{Amsterdam}
\country{The Netherlands}
}
\email{something}

\author{Daan Odijk}
\affiliation{
\institution{RTL} 
\city{Amsterdam}
\country{The Netherlands}
}
\email{something}

\author{Maarten de Rijke}
\affiliation{
\instutute{University of Amsterdam \& Ahold Delhaize}
\city{Amsterdam}
\country{The Netherlands}
}
\email{m.derijke@uva.nl}

% !TEX root =  ./main.tex 

\copyrightyear{2021}
\acmYear{2021}
\setcopyright{rightsretained}
\acmConference[SIGIR '21]{Proceedings of the 44th International ACM SIGIR Conference on Research and Development in Information Retrieval}{July 11-15, 2021}{Montr{\'{e}}al, Canada}
\acmBooktitle{Proceedings of the 44th International ACM SIGIR Conference on Research and Development in Information Retrieval (SIGIR '21), July 11-15, 2021, Montr{\'{e}}al, Canada}
\acmPrice{}
\settopmatter{printfolios=false}

\usepackage{xcolor}
\newcommand\todo[1]{\textcolor{red}{TODO : #1}}
\newcommand\doubt[1]{\textcolor{orange}{DOUBT : #1}}
% \newcommand\todo[1]{} % uncomment to hide comments
% \newcommand\doubt[1]{} % uncomment to hide comments
\usepackage{amsmath}
\usepackage{dsfont}
\usepackage{color}
\author{Gabriel Bénédict}
\date{\today}
\title{}
\hypersetup{
 pdfauthor={Gabriel Bénédict},
 pdftitle={},
 pdfkeywords={},
 pdfsubject={},
 pdfcreator={Emacs 26.1 (Org mode 9.1.14)}, 
 pdflang={English}}
\begin{document}

% https://sigir.org/sigir2021/checklist-to-strengthen-an-ir-paper/
% Things that strengthen an IR paper: recommendations from the Program Chairs
% Presentation
% The paper’s motivation and the potential impact of the addressed problem are discussed.
% The paper’s original contributions (i.e. the delta over prior art) are clearly stated.
% The paper’s claims are properly scoped and supported.
% The paper clearly describes what was done and what was not.
% The choices made in each step of the research are justified (the why’s).
% The results are presented effectively in appropriate format.
% Good discussion accompanies the results.
% Experimentation (if applicable)
% The experimental design and its scale are appropriate.
% In comparative studies, appropriate baselines are used.
% The experimental results are reliable and generalizable.
% The evaluation methods employed are in line with the research questions.
% Statistical analysis is performed and reported appropriately.
% Sufficient details (with data and code where appropriate) are provided to help other researchers assess and reproduce the experiments.


% bibliographystyle:apa

% #+LATEX_HEADER: \documentclass[sigconf,natbib,screen=true,review=true,anonymous]{acmart}

% #+LATEX_HEADER: \usepackage{aistats2018}
% #+LATEX_HEADER: \usepackage{natbib}

% #+Latex_header: \usepackage{hyperref}
% #+Latex_header: \usepackage{mathtools}  % amsmath with extensions
% #+Latex_header: \usepackage{amsfonts}  % (otherwise \mathbb does nothing)
% #+Latex_header: \usepackage{amssymb}
% highlight sections
% https://latex.org/forum/viewtopic.php?t=27521

% #+LATEX_HEADER: \usepackage{xcolor}
% #+LATEX_HEADER: \usepackage[explicit]{titlesec}
% #+LATEX_HEADER: \definecolor{Blue}{rgb}{0,0,0.8}
% #+LATEX_HEADER:
% #+LATEX_HEADER: \titleformat{\subsection}
% #+LATEX_HEADER:    {\normalfont\large\bfseries\color{Blue}}% format
% #+LATEX_HEADER:    {}% label
% #+LATEX_HEADER:    {0pt}% sep
% #+LATEX_HEADER:    {\titlerule\newline\llap{\thetitle. }#1}% before code
% #+LATEX_HEADER:    [{\titlerule[0.4pt]}]% after code
% #+LATEX_HEADER:
% #+LATEX_HEADER: \titlespacing{\subsection}
% #+LATEX_HEADER:    {25pt}% left
% #+LATEX_HEADER:    {0pt}% before sep
% #+LATEX_HEADER:    {\baselineskip}% after sep


\title{Loss framework for uni-instance multiclass multilabel prediction with varying number of labels}


% #+BEGIN_LATEX latex
% \twocolumn[

% \aistatstitle{Top-\( \kappa \) : top k multilabel prediction with varying k}

% \aistatstitle{DynaMu Loss : a loss framework for multilabel prediction with varying number of groundtruth labels}
% # a framework for multilabel prediction with varying amount of labels [Varying Label Amounts Prediction] or top k with varying k

% \aistatsauthor{ Author 1 \And Author 2 \And  Author 3 }

% \aistatsaddress{ Institution 1 \And  Institution 2 \And Institution 3 } ]
% #+END_LATEX

\begin{abstract}
Multilabel classification is a common task in text, image or video (scene) prediction.
\end{abstract}


\keywords{Keyword; Keyword; Keyword}

\maketitle

\acresetall

\section{introduction}
\label{sec:org92bd4fa}

As neural network models are able to learn increasingly abstracter representations via deeper networks, representation learning and self-supervision, it might be reasonable to expect that, thanks to their conferred broader understanding of the world, they get better at predicting more abstract labels. Beyond objects types, face recognition, expressions, neural networks might be able to predict genres/categories \todo{other things as well?} of text, image and sound. While researchers are working hard at building neural networks with very high level of understanding in the embedding space, there seems to be few research on developing loss functions that are adapted for these higher level concepts in the output space.

Although multilabel binary prediction (commonly referring to mutually inclusive labels) is a task thoroughly covered in existing litterature, there does not seem to exist a framework that deals with different amounts of positive labels in the groundtruth. For example, a scientific journal can be tagged as \emph{machine learning} and \emph{economics}, or a movie can be tagged as \emph{romance} and \emph{comedy}. These instances might as well be assigned only one tag in the groundtruth, or many more within the possible tags (classes).

\begin{figure}[htbp]
\centering
\includegraphics[width=.9\linewidth]{./tree/Tree.pdf}
\caption{\label{fig:tree}
"multiclass" nomenclature}
\end{figure}

Before, exploring the subject further, we will use Figure \ref{fig:tree} to disambiguate the terminology used in this research. There seems to exist a concensus over the terms multiclass and multilabel learning, meaning respectively mutually exclusive and mutually inclusive labels \todo{source}. Multilabel can therefore be seen as a subdomain of multiclass learning, where more than one class can be true for the same example. Within multilabel training, we introduce the distinction between multi-instance multilabel (e.g. \cite{multiInstance}) and uni-instance multilabel. The former refers to tasks where elements within each example can be singled-out (objects in an image or expressions in a text) and assigned one or more labels. In the countrary, this paper focusses on uni-instance multilabel training (sparse occurences of the term holistic can be found in the litterature to describe this phenomenon for image \cite{holisticImageDescriptors,holisticLungs} and a recent video dataset \cite{holisticVideoData} \todo{read these}), more specifically with varying label counts. To the best of our knowledge, there are few existing representatives of that type of labelling task in the literature. \todo{cite more milestone examples for each category.} \todo{delta with hierarchical label learning}

The particularity of tasks like scientific paper tagging or movie genre classification is that it remains unclear what elements in an image/video or text can be singled out as predictive of a particular tag/genre. Rather, a complex interaction between these elements in the feature space steer the predictions. For example, the sole mention of the term "machine learning" in a paper should not be a sufficient condition to tag it as such. Instead, one could expect from the publisher to get acquainted with the paper enough to determine wether the research is a worthwhile contribution or application of \emph{machine learning} to deserve the tag. This involves thorough understanding of the proposed method and background knowledge on state-of-the-art methods. An analogous argument can be made for movie genre classification for movie posters.

However, if elements in an image/text can be singled out as predictive of a single tag, the problem reverts back to predicting with the a priori knowledge of the existence of only one true label (i.e. multi-instance multilabel learning).  The reason for distanciating singling-out from uni-instance labels, is that it has been shown that as soon as singling-out is possible, models that work on instances are more accurate \todo{rewrite this paragraph and sources}. The singled-out elements can be subsets of the original feature space (typically in object detection like with the COCO dataset  \cite{COCO} or the Amazon Rainforest Dataset\footnote{Available at https://www.kaggle.com/c/planet-understanding-the-amazon-from-space} \todo{others}). Similarly, recent research has shown that the singled-out elements can be located in the abstract representations (embeddings) of the feature set and might individually predict a single true label (like GPT-3 \todo{source}) \todo{more examples}. This might also carry prospects of generalizability of the model \cite{generalization} \todo{elaborate}. 

But for now, in certain retrieval tasks such as scientific journal tagging, the effect of sub-entities (either expressions in the text or single features in the embedding space) on the prediction of each label remains hard to assess. Instead we propose uni-instance (sometimes referred to as holistic) multilabel learning for varying amount of labels, with a focus on custom loss functions.

To allow the use of existing diffentiable loss fonctions, previous research papers tend to reframe the problem into either (I) a multi-instance multiclass (as described above, with the COCO dataset as an example of isolation of features \cite{COCO}), (II) uni-instance multiclass prediction (III) uni-instance multilabel prediction with fixed label count (IV) uni-instance multilabel prediction with varying label count with post-training thresholding (V) redefine backpropagation for multilabel prediction \cite{multilabelBackprop} (VI) multitask learning \cite{multitaskLabel} (VII) custom loss function \cite{tencent}. This order reflects in ascending order how close modelling seem to fit the original task, which remains uni-instance multilabel learning with varying amounts of labels. \doubt{group them}

Common loss functions such as cross-entropy loss (for mutually inclusive labels) or multinomial logit loss (for mutually exclusive labels) deliver predictions on the unit interval. Thresholding the output to assess the performance of the model against the groundtruth can be done after training for (I), (II), (III) and (IV). \todo{give a very sound reason as to why we'd rather not do things post-training and rather at training-time}. Problem formulations (V), (VI) and (VII) suggest a solution at training time. We think that a custom loss function (VII) is the best alternative. \todo{explain why}

In a number of retrieval tasks, a model's out of sample accuracy is measured on metrics such as AUROC, F1 score, etc. These reflect an objective catered towards evaluating the model over an entire ranking. Due to to lack of differentiability, these metrics cannot be directly used as loss functions at training time (in-sample). A seminal study \cite{optimizableLosses} derived a general framework for deriving decomposable surrogates to some of these metrics. We propose our own decomposable F1 surrogate tailored for the problem at hand.

We first propose a general mathematical formulation of uni-instance multilabel learning for varying amount of groundtruth labels. The generalization encompasses different levels of complexity, from the classical cross-entropy loss up to the proposed loss function. \emph{sigmoidF1} is a F1 score surrogate which allows to optimize for label prediction and count simultanuously in a single task and is robust to outliers. It delivers more precise predictions than the current state-of-the-art on several different metrics, accross text and image related tasks. \emph{sigmoidF1} and its adaptive \emph{SadF1} and Bayesian \emph{SBayesF1} counterparts are benchmarked against loss functions commonly used in multilabel learning and others tailored specifically to the uni-instance multilabel with varying number of labels setting.

\section{Building up on losses}
\label{sec:org6086cea}

Multi-label learning can be divided into two major fields: \emph{problem transformation} and \emph{algorithm adaptation} \cite{multilabelReview}. In the former case, multilabel classification is reframed as a binary, multiclass classification or label ranking problem. In the latter, one tries to adapt multiclass algorithms to the problem. The current endeavour focusses on \emph{algorithm adaptation}.


For the purpose of \emph{problem transformation}, we define \(\mathcal{L}_{\text {multiclass}}\), a class of loss functions that minimize predictions in relative terms. Binary cross-entropy, logit and their variants such as focal loss or hinge loss (deemed unstable \cite{focalLoss}) are common choices when it comes to multiclass prediction. Cross-entropy loss can be formulated as \(\mathcal{L}_{\text {CE}}=-\sum \log \left(p_{i}\right)\) . Note that minimizing binary cross-entropy is equivalent to maximizing for log-likelihood \cite[Section 4.3.4]{Bishop}. More generally, the \emph{problem transformation} formulation amounts to minimizing the loss on a class of neural networks, such that

\begin{equation}
\underset{\mathcal{L}_{\text {multiclass}}} {\min} \mathcal{F}\left(\cdot ; \Theta; \mathcal{L}_{\text {multiclass}} (\mathbf{y}, \hat{\mathbf{y}}) \right),
\end{equation}

In the context of \emph{algorithm adaptation}, where the number of positive labels in the groundtruth is unknown a priori, we aim to both obtain a propensity of each label being true and a prediction of the number of true labels: 

\begin{equation}
\underset{\mathcal{L}_{\text {multiclass}}, \mathcal{L}_{\text {count}}} {\min} \mathcal{F}\left(\cdot ; \Theta; \mathcal{L}_{\text {multiclass}} (\mathbf{y}, \hat{\mathbf{y}}) + \lambda \mathcal{L}_{\text {count}} (\mathbf{n}, \hat{\mathbf{n}})\right),
\end{equation}

where \(n_i = \sum_j \mathds{1}_{\mathbf{y_i^j} = 1}\) is the count of positive labels per example. We thus impose a constraint for the retrieval of label counts. For example, a cross-entropy loss surrogate would penalize for the number of wrongly predicted labels \(\mathcal{L}_{\text {CE+N}}= \mathcal{L}_{\text {CE}} + \lambda (\sum tp / \sum p)\), with \(t p=\sum_{i \in Y^{+}} \mathds{1}_{\mathbf{p_i} \geq b}\) and \(b\) a threshold to be defined. \todo{tencent loss}.

This formulation is most straightfoward but suffers from higher parametrization and the lack of modelling of the interactions between label counts and label prediction. To mitigate these issues, we propose a unified loss formulation, namely

\begin{equation}
\underset{\mathcal{L}_{\text {multitag}}} {\min} \mathcal{F}\left(\cdot ; \Theta; \mathcal{L}_{\text {multitag}} (\mathbf{y}, \hat{\mathbf{y}}, \mathbf{n}, \hat{\mathbf{n}}) \right),
\end{equation}

Although predictions and counts explicitly appear in that formulation, \(\mathcal{L}_{\text {multitag}}\) can optimize for both metrics implicitely (see proposed \emph{sigmoidF1} below).


\todo{look at YOU ONLY TRAIN ONCE: LOSS-CONDITIONAL TRAINING OF DEEP NETWORKS}

\todo{cite stat learning}   \cite[p. 308-310]{statLearning}


% * our contribution

% In order to verify our hypotheses, we use multilabel examples, where it is arguably hard to distinguish which elements is predictive of the label. For example, movie posters where the whole context is important and not just facial expressions, title font.

\section{related work}
\label{sec:org69a4fe8}

\todo{look at [[https://www.sciencedirect.com/topics/computer-science/extractive-summarization][extractive summarization]]}

This section will be guided by the previous section's formulation of the multitags problem, we will therefore focus on \emph{algorithm adaptation}, \emph{metrics as losses} and \emph{dynamic thresholding}.

\subsection{algorithm adaptation}
\label{sec:orge2a628b}

Early representatives of \emph{algorithm adaptation} stem from heterogenous domains of machine learning. Multi-Label k-Nearest Neighbors \cite{ML-KNN}, Multi-Label Decision Tree \cite{ML-DT}, Ranking Support Vector Machine \cite{multilabelSVM} and Backpropagation for Multi-Label Learning \cite{multilabelBackprop}. More recently, two papers introduced the idea of multitask learning for \emph{label prediction} and \emph{label count prediction} for text (ML\(_{\text{NET}}\)) \cite{multitaskLabel} and image \cite{multitaskLabelImages} data. The latter research is loosely catered towards object detection (although not formally presented as such) and is thus out-of-scope: elements in a picture are predicted that tend to be unilabel as defined by the groundtruth (e.g. cat, flower, vase, person, bottle etc.).

\subsection{metrics as losses}
\label{sec:org00ec103}

Often, machine learning post-training evaluation metrics (e.g. AUROC, F1) are not differentiable. There are motivations \todo{which motivations} for optimizing a model directly on a metric at training time. A general framework for AUC, AUROC and F1 is presented in \cite{optimizableLosses}, but the proposed F1 surrogate remains short of being explicitly derived for stochastic gradient descent. \todo{check again with the authors if I can't get inspired from their work}. Recently, a similar work has been proposed to train a Convolutional Neural Network (CNN) from scratch with a few millions of images and hundreds of labels specifically for multilabel tasks \cite{tencent}. This task is loosely related to object detection, similarly to \cite{multitaskLabelImages} mentioned in the previous paragraph.


% in reformulating loss functions to accomodate sparsity in the data, to optimize directly for the metric at hand or to do thresholding posthoc (see movie posters).

\subsection{dynamic thresholding}
\label{sec:org2faaace}

\emph{dynamic thresholding} accross classes or examples is an issue as soon as the number of labels to predict is unknown. Certain variants of cross-entropy loss accomodate imbalanced label data  \cite{focalLoss}, but remain agnostic towards the number of labels to predict. Solutions have been tailored to that end, starting with determining an ideal global \emph{threshold} depending on use-cases \cite{threshForF1}, or per-class-thresholding after training \cite{moviePosters} and eventually abstracting the threshold away via a \emph{soft-F1} measure \cite{softF1} \todo{say more about this method}. In the latter two cases, the task is to predict genre from movie posters.

\begin{figure}[htbp]
\centering
\includegraphics[width=.9\linewidth]{./images/knee.png}
\caption{\label{fig:knee}
ordered per-label cross-entropy predictions for each example (each grey line) with the median (orange) and IQR (green \& blue) over all examples. Determining a global threshold can be related to visually finding the "knee" in that median curve (dotted line)}
\end{figure}

\todo{nicer plot on another dataset (this is from RTL)}

The proposed method is positioned in the lineage of \emph{algorithm adaptation}, using \emph{metric as losses} and allowing for \emph{dynamic thresholdig}. 

% We propose a dynamic thresholding mechanism auto-tuned at training time.


% ** weak labels
% (unsure the labels are correct)

% - https://people.cs.pitt.edu/~kovashka/ye_zhang_kovashka_iccv2019_cap2det.pdf


% ** implementations

% *** movies

%  [[https://www.analyticsvidhya.com/blog/2019/04/build-first-multi-label-image-classification-model-python/][movie posters with classes]].

%  They have movie titles in them

% *** pretrained resnet on multilabel

%  https://github.com/Tencent/tencent-ml-images

% What happens when using a Resnet pretrained on multilabels

% *** soft F1 score loss

%  https://github.com/ashrefm/multi-label-soft-f1

% https://www.analyticsvidhya.com/blog/2019/04/build-first-multi-label-image-classification-model-python/



% /Optimizing directly for macro F1: By introducing the macro soft-F1 loss, we could train the model to directly increase the metric we care about: the macro F1-score @ threshold 0.5. We could clearly observe the alignment during training and evaluation on successive epochs. When using this loss, we do not have to tune the decision threshold any more. Imagine a multi-label classification system with hundreds of labels, how unstable the system will be if we have to continuously update the optimal threshold for each label. The macro soft-F1 loss comes to the rescue. By using it, we can keep all thresholds fixed at 0.5 and still get an optimal performance from the training process./

\section{Sigmoid F1 loss}
\label{sec:org216c136}

For a class of multilayer perceptron \(\mathcal{F}(\cdot ; \Theta): \mathcal{X} \rightarrow \mathcal{Y}\), we consider a special case, where \(\mathbf{x} = \{x_1, ..., x_n\}\). Each observation is attributed one or more classes out of a label set \(\mathbf{l} = \{l_1, ..., l_c\}\). Labels \(y_{i}^{j}\) are available for each observation \(i\) and class \(j\). 

For each observation \(i\), label class probabilities can be defined based on predictions as

\todo{check this formula}

\begin{equation}
\mathbf{p}_{i}=\left\{\begin{array}{ll}\hat{\mathbf{y}} & \text { if } y=1 \\ 1-\hat{\mathbf{y}} & \text { otherwise }\end{array}\right.
\end{equation}

Let \(tp\) and \(fp\) be number of true and false positives respectively. It is necessary to define a bound \(b\), at which a prediction is dichotomized:

\begin{equation}
\label{eq:conf}
 t p=\sum_{i \in Y^{+}} \mathds{1}_{\mathbf{p_i} \geq b} \quad f p=\sum_{i \in Y^{-}} \mathds{1}_{\mathbf{p_i} \geq b} \quad fn = \sum_{i \in Y^{+}} \mathds{1}_{\mathbf{p_i} < b}
\end{equation}

\(\mathds{1}_{\mathbf{p_i} \geq b}\), \(\mathds{1}_{\mathbf{p_i} < b}\) are thus the count of positive and negative predictions at threshold \(b\), 

We also define precision and recall

\begin{equation}
\begin{aligned} P &=\frac{t p}{t p+f p} \\ R &=\frac{t p}{t p+f n}=\frac{t p}{\left|Y^{+}\right|} \end{aligned}
\end{equation}

We can then define \(F_\beta\), which can be expressed as the effectiveness of retrieval with respect to a user who attaches \(\beta\) times as much importance to recall than precision \cite{informationRetrieval}.

\doubt{maybe ignore $F_\beta$ and only mention $F_1$}

\begin{equation}
F_{\beta}=\left(1+\beta^{2}\right) \frac{P \cdot R}{\beta^{2} P+R}
\end{equation}

Or equivalently:

\begin{equation}
\begin{aligned} F_{\beta} &=\left(1+\beta^{2}\right) \frac{t p}{\left(1+\beta^{2}\right) t p+\beta^{2} f n+f p} \\ &=\left(1+\beta^{2}\right) \frac{t p}{\beta^{2}|Y+|+t p+f p} \end{aligned}
\end{equation}

Given the presence of the step indicator function \(\sum \mathds{1}_{\mathbf{p_i} \geq b}\), \(F_\beta\) is not differentiable for gradient based methods. One way of surpassing that problem is to use a smooth surrogate.

\subsection{soft F1 score}
\label{sec:org6f0f270}

It is possible define a \emph{soft F1} score \cite{softF1} \doubt{can we cite a Medium post?} with smooth confusion matrix entries (i.e. \(tp\), \(fp\) and \(fn\) are not natural numbers anymore):

$$
\overline{tp}=\sum \hat{\mathbf{y}} \odot \mathbf{y} \quad \overline{fp} = \sum \hat{\mathbf{y}} \odot (\mathbf{1}- \mathbf{y}) \quad \overline{fn} = \sum (\mathbf{1} - \hat{\mathbf{y}}) \odot \mathbf{y}
$$

\begin{equation}
\mathcal{L}_{\text {softF1}}= \frac{\overline{tp}}{2 \overline{tp}+ \overline{fn}+ \overline{fp}}
\end{equation}

\(tp\), \(fp\) and \(fn\) are now replaced by rough surrogates, this method has the advantage of 

% /softF1/ is
% $$\mathcal{L}_{\text {Pred}}=\sum_{i, j}\left(\mathbf{y}_{i j}-\hat{\mathbf{y}}_{i j}\right)^{2}$$

\subsection{sigmoidF1 score}
\label{sec:orgc404c88}

We define \emph{sigmoidF1}, inspired by the \emph{Maximum F1-score criterion} for automatic mispronounciation detection \cite{sigmoid}. Whereas a sigmoid function \(S(u)\)

\begin{equation}
S(u; \beta, \eta)=\frac{1}{1+\exp (-\beta (u + \eta))},
\end{equation}

with \(\beta\) and \(\eta\) tunable parameters for slope and offset respectively. Higher \(\beta\) results in steeper slope at the center of the sigmoid and thus more stringent thresholding. At its extreme, \(lim_{\beta\to\infty} S(u; \beta, \eta)\) corresponds to the step function used in Equation \ref{eq:conf}. with \(S(u)\), the confusion matrix entries then become

$$
\widetilde{tp}=\sum S(\hat{\mathbf{y}}) \odot \mathbf{y} \quad\widetilde{fp}= \sum S(\hat{\mathbf{y}}) \odot (\mathbf{1} - \mathbf{y}) \quad \widetilde{f n}= \sum (\mathbf{1} - S(\hat{\mathbf{y}})) \odot \mathbf{y}
$$

And thus

\begin{equation}
\mathcal{L}_{\text {softF1}}= \frac{\widetilde{tp}}{2 \widetilde{tp}+ \widetilde{fn}+ \widetilde{fp}}
\end{equation}

\doubt{mention smooth hinge loss} \cite{smoothHinge}

For \emph{sigmoidF1} \(\beta\) and \(\eta\) are tuned globally as hyperparameters. \emph{SAdF1} (Sigmoid Adaptive F1), is an alternative where \(\beta\) is first set to a relatively low value and increased after each epoch. This way, a loose threshold first allows Stochastic Gradient Descent (SGD) to broadly scan the parameter space accross several local minima, before narrowing parameter search down to a promissing region (similarily to adaptive learning rates).

\emph{SBayesF1} (sigmoid Bayes F1) replaces point estimates for \(\beta\) and \(\eta\) with posterior distribution estimates. 

\begin{equation}
S(u_i) = \frac{1}{1+\exp (-\beta_i (u_i + \eta_i))}
\end{equation}

% $$ \beta_i \tilde \mathcal{N} (0, \sigma^{2}_{\beta}) $$

% $$ \eta_i \tilde \mathcal{N} (0, \sigma^{2}_{\eta}) $$

\(\beta_i\) and \(\eta_i\) are estimated with MCMC at training time of the neural network. They are therefore implicitely allowed to vary across examples.

\todo{try SadF1 and SBayesF1 in practice}


\subsection{Robustness}
\label{sec:org85a2343}


Similarly to the focal loss \cite{focalLoss}, sigmoidF1 loss deals with class imbalance, robustness to outliers.

\todo{statistical robustness assessment}



\subsection{Evaluation Metrics}
\label{sec:org0e9813d}

The metrics described below are a result of a survey of different common practices for measuring accuracy of multilabel prediction. When true positives and false positives are used, recall that \(t p=\sum_{i \in Y^{+}} \mathds{1}_{\mathbf{p_i} \geq b}\) and \(f p=\sum_{i \in Y^{-}} \mathds{1}_{\mathbf{p_i} \geq b}\), and thus a threshold \(b\) must be set. When \(b = 0.5\), as is commonly done [SOURCE HERE], a risk remains that a lot of examples remain without predictions.

Extending \(F_1\) to multi-class binary classification amounts to deciding wether to un/pool classes.
In a first pooled iteration, micro \(F_1\) [SOURCE HERE] equates to creating a single 2x2 confusion matrix for all classes:
$$F_1^{micro} = \frac{\sum tp_c}{2 \sum tp_c + \sum fn_c + \sum fp_c} \quad for \quad c \in C$$

Macro \(F_1\) \cite{threshForF1} amounts to creating one confusion matrix per class:

$$F_1^{macro} = \frac{1}{c} \sum_{j=1}^c F_1$$

\doubt{Do we need to justify optimizing for an F1 surrogate at training time and to then use F1 itself as a metric?}
% $$F_1^{macro} = \frac{\sum tp_c}{2 \sum tp_c + \sum fn_c + \sum fp_c} \quad for \quad c \in C$$

Weighted macro \(F_1\) \todo{find source} is similar but includes weighing to account for class imbalance, i.e. weighing each class by the number of groundtruth positives.

$$F_1^{weighted} = \frac{1}{c} \sum_{j=1}^c n_j F_1 \quad where \quad n_j = \sum_i \mathds{1}_{\mathbf{y_i^j} = 1}$$

% $$F_1^{weighted} = \frac{\sum tp_c}{2 \sum tp_c + \sum fn_c + \sum fp_c} \quad for \quad c \in C$$

Accuracy is the overall fraction of correctly predicted labels \cite{threshForF1}:

$$
A c c=\frac{t p+t n}{t p+t n+f p+f n}
$$

% - 'samples':
% Calculate metrics for each instance, and find their average (only meaningful for multilabel classification where this differs from accuracy_score).

% $$F_1^{micro} = \frac{\sum tp_c}{2 \sum tp_c + \sum fn_c + \sum fp_c} \quad for \quad c \in C$$


\subsection*{{\color{red}\bfseries\sffamily TODO} compare to  \cite{lossComp}}
\label{sec:orgefab61e}
\clearpage

\section{datasets}
\label{sec:org49fdd2d}

sigmoidF1 is tested across different modalities, namely image, video, sound and text, with a focus on text: the most comparable research was on text data.

% \doubt{optional paragraph}
% In light of the problem definition leading to the sigmoidF1 framework in the introduction and in order to clearly delimit the proposed method, following are a few datasets that are not suitable for the task.


Among the three datasets used for benchmarking ML-NET \cite{multitaskLabel}, a cancer hallmark dataset is of multi-instance multilabel nature \cite{cancerHallmarks} \footnote{Available at https://www.cl.cam.ac.uk/&sim;sb895/HoC.html}: the research clearly describe a process of annotating several expressions within paper abstracts. The remaining two datasets for chemical exposure \cite{chemExposure} \footnote{Available at https://figshare.com/articles/Corpus_and_Software/4668229} and diagnosis codes assigment \cite{diagnosisCode} \footnote{Available at https://physionet.org/works/ICD9CodingofDischargeSummaries}, seem to fit to the entity wide multilabel definition but have a strong hierarchical nature. Although slightly out-of-scope, the three datasets above will be used for benchmarking, since they were used to test ML-NET, which is the state-of-the-art in \emph{algorithm adaptation} for text to the best of our knowledge.

For a broader scope in learning for text data, we also use the newly created \emph{Arxiv dataset} \footnote{Available at https://www.kaggle.com/Cornell-University/arxiv} with data on abstracts of 1.7 million open source articles and their categories (suitably mutually inclusive and of varying count per example).

In the vision domain, a dataset of movie posters \footnote{Labels available at https://tinyurl.com/y7ydyedu and prescraped images from IMDB at https://tinyurl.com/y7lfpvlx} and their genre is used. Similarly, labels are mutually inclusive and of varying count per example. It is arguable that is hard to single out elements in the image of a poster that define the genre of a movie. Rather it might be a combination of the title font, the background image, the presence of actors and specific objects such as cars, weapons etc. 




\todo{I removed all jpg's that are empty in the prescraped data. I could try to scrape the posters myself to see if I get more}

Another recently created dataset was made available for \emph{Large Scale Holistic Video Understanding} \cite{holisticVideoData} \footnote{Available at https://github.com/holistic-video-understanding/HVU-Dataset}, as defined in the introduction.

% Cancer can be described according to its complexity with different principles, named hallmarks cite:cancerHallmarks. A corpus of 1580 PubMed abstracts are manually annotated for 10 hallmarks. This is a multi-instance labelling task and will therefore not be used here.

% [[./images/cancerHallmarksAnnotation.jpg]]

% - Multilabel classification for text cite:toxicComments

% - Scenery dataset for images cite:dataScenery.

\todo{this is an ambitious number of datasets. Add longer description of each dataset, depending on which ones I keep: sample size, number of classes etc. see utils here: https://github.com/ashrefm/multi-label-soft-f1}

\doubt{cite Kaggle datasets formally instead of using links: https://www.kaggle.com/data/46091}

\doubt{add a music genre classification dataset, for which Vincent Koops at RTL could help train}

\newpage

\section{Experimental Results}
\label{sec:org18c7694}

varying b in the sigmoid function as if it is an adaptive learning rate \todo{actually try it out}.

one b per class

if we consider \(b\) and \(c\) to be probabilistic, we can then use tensorflow probability to assess their distribution

the batch size has to be relatively large (i.c. 256), in order for meaningful F1 surrogates to be calculated.



\textbf{movie posters (CNN)}

\begin{array}{cccccc}\hline Loss  & \rotatebox[origin=c]{270}{macroF @ 0.5} & \rotatebox[origin=c]{270}{microF1 @ 0.5} & \rotatebox[origin=c]{270}{weightedF1 @ 0.5} & \rotatebox[origin=c]{270}{Precision @ 0.5} & \rotatebox[origin=c]{270}{Recall @ 0.5}\\ 
\hline \mathcal{L}_{\text {CE}} & 0.057 & 0.200 & 0.159 & 0.106 & 0.106 \\ 
\mathcal{L}_{\text {FL}} & 0.055 & 0.192 & 0.154 & 0.115 & 0.115 \\
\mathcal{L}_{\text {CE+N}} & 0 & 0 & 0 & 0 & 0 \\
\mathcal{L}_{\text {CE+T}} & 0 & 0 & 0 & 0 & 0 \\
\mathcal{L}_{\text {macroSoftF1}} & 0.132 & 0.323 & 0.280 & 0.105 & 0.105 \\
\mathcal{L}_{\text {sigmoidF1}} & \mathbf{0.117} & \mathbf{0.240} & \mathbf{0.263} & \mathbf{0.103} & \mathbf{0.103} \\
\hline\end{array}

\textbf{Arxiv (distillBERT)}

\begin{array}{ccccc}\hline \text { Metric } & \mathcal{L}_{\text {CE}} & \mathcal{L}_{\text {FL}} & \mathcal{L}_{\text {CE+N}} & \mathcal{L}_{\text {CE+T}} \\ 
\hline P(\%) & 0 & 0 & 0 & 0 \\ 
R(\%) & 0 & 0 & 0 & 0 \\
F_{1}(\%) & 0 & 0 & 0 & \mathbf{0} \\
\hline\end{array}


\textbf{Cancer hallmark (distillBERT)}

\begin{array}{ccccc}\hline \text { Metric } & \mathcal{L}_{\text {CE}} & \mathcal{L}_{\text {FL}} & \mathcal{L}_{\text {CE+N}} & \mathcal{L}_{\text {CE+T}} \\ 
\hline P(\%) & 0 & 0 & 0 & 0 \\ 
R(\%) & 0 & 0 & 0 & 0 \\
F_{1}(\%) & 0 & 0 & 0 & \mathbf{0} \\
\hline\end{array}

\textbf{Chemical exposure (distillBERT)}

\begin{array}{ccccc}\hline \text { Metric } & \mathcal{L}_{\text {CE}} & \mathcal{L}_{\text {FL}} & \mathcal{L}_{\text {CE+N}} & \mathcal{L}_{\text {CE+T}} \\ 
\hline P(\%) & 0 & 0 & 0 & 0 \\ 
R(\%) & 0 & 0 & 0 & 0 \\
F_{1}(\%) & 0 & 0 & 0 & \mathbf{0} \\
\hline\end{array}

\textbf{Chemical exposure (distillBERT)}

\begin{array}{ccccc}\hline \text { Metric } & \mathcal{L}_{\text {CE}} & \mathcal{L}_{\text {FL}} & \mathcal{L}_{\text {CE+N}} & \mathcal{L}_{\text {CE+T}} \\ 
\hline P(\%) & 0 & 0 & 0 & 0 \\ 
R(\%) & 0 & 0 & 0 & 0 \\
F_{1}(\%) & 0 & 0 & 0 & \mathbf{0} \\
\hline\end{array}

\textbf{simulated data}

\begin{array}{ccccc}\hline \text { Metric } & \mathcal{L}_{\text {CE}} & \mathcal{L}_{\text {FL}} & \mathcal{L}_{\text {CE+N}} & \mathcal{L}_{\text {CE+T}} \\ 
\hline P(\%) & 0 & 0 & 0 & 0 \\ 
R(\%) & 0 & 0 & 0 & 0 \\
F_{1}(\%) & 0 & 0 & 0 & \mathbf{0} \\
\hline\end{array}

\section{conclusion}
\label{sec:org98f051a}

\textbf{Shortcomings}

it is debatable wether any task is intrinsincly multilabel and wether the image / text cannot be decomposed in parts that are single labelled.

not long training and small models, but aibility to demonstrate the statement anyways.

\textbf{Results}

In this paper we defined a new problem in deep learning for mulitple modalities that harness the current advances in abstract representation of the input space. A general loss framework is proposed to locate that solution within the existing multiclass multilabel losses and a specific loss function is formulated. \emph{sigmoidF1} achieves significantly results for different F1 values on all datasets.

\textbf{Future work}

Apply the loss function to more sophisticated neural network architectures that use F1 score as an evaluation metric such as AC-SUM-GAN \cite{AC-SUM-GAN}.

This model can be adapted for hiarchical multilabel classification or active learning (for both see \cite{activeLearningMultiLabel}).

Combine the proposed loss functions with representation learning \cite{unsupervisedImage,highResRepresentation} or self-supervised learning, in order to model abstract relationships between the labels.

adapt to \emph{extreme} multilabel prediction \cite{extremeMultilabelText}


\begin{acks}}
 This work was supported by many people.
 All content represents the opinion of the authors, which is not necessarily shared or endorsed by their respective employers and/or sponsors.
\end{acks}

\bibliographystyle{ACM-Reference-Format}


\bibliography{multilabel}
\end{document}
