% !TEX root = ../main.tex

\section{Experimental Results}
\label{sec:orgc23a664}

varying b in the sigmoid function as if it is an adaptive learning rate \todo{actually try it out}.

one b per class

if we consider \(b\) and \(c\) to be probabilistic, we can then use tensorflow probability to assess their distribution

the batch size has to be relatively large (i.c. 256), in order for meaningful F1 surrogates to be calculated.



\begin{table}
\caption{Movie posters (CNN). \mdr{Explain what we see.}}
\centering
\begin{tabular}{l ccccc}
\toprule 
Loss  & \rotatebox[origin=c]{90}{macroF @ 0.5} & \rotatebox[origin=c]{90}{microF1 @ 0.5} & \rotatebox[origin=c]{90}{weightedF1 @ 0.5} & \rotatebox[origin=c]{90}{Precision @ 0.5} & \rotatebox[origin=c]{90}{Recall @ 0.5}\\ 
\midrule
$\mathcal{L}_{\text {CE}}$ & 0.057 & 0.200 & 0.159 & 0.106 & – \\ 
$\mathcal{L}_{\text {FL}}$ & 0.055 & 0.192 & 0.154 & 0.115 & – \\
$\mathcal{L}_{\text {CE+N}}$ & 0 & 0 & 0 & 0 & 0 \\
$\mathcal{L}_{\text {CE+T}}$ & 0 & 0 & 0 & 0 & 0 \\
$\mathcal{L}_{\text {macroSoftF1}}$ & 0.132 & 0.323 & 0.280 & 0.105 & – \\
$\mathcal{L}_{\text {sigmoidF1}}$ & \textbf{0.117} & \textbf{0.240} & \textbf{0.263} & \textbf{0.103} & \textbf{–} \\
\bottomrule
\end{tabular}
\end{table}


\begin{table}
\caption{Arxiv (distillBERT 2020), frozen pretrained weights 100 epochs, min-label-thresh: 1000}
\centering
\begin{tabular}{l ccccc}
\toprule
Loss  & \rotatebox[origin=c]{90}{macroF @ 0.5} & \rotatebox[origin=c]{90}{microF1 @ 0.5} & \rotatebox[origin=c]{90}{weightedF1 @ 0.5} & \rotatebox[origin=c]{90}{Precision @ 0.5} & \rotatebox[origin=c]{90}{Recall @ 0.5}\\ 
\midrule
$\mathcal{L}_{\text {CE}}$ & 0.093 & 0.106 & 0.106 & 0.096 & – \\ 
$\mathcal{L}_{\text {FL}}$ & 0 & 0 & 0 & 0 & 0 \\
$\mathcal{L}_{\text {CE+N}}$ & 0 & 0 & 0 & 0 & 0 \\
$\mathcal{L}_{\text {CE+T}}$ & 0 & 0 & 0 & 0 & 0 \\
$\mathcal{L}_{\text {macroSoftF1}}$ & 0.077 & 0.088 & 0.087 & 0.100 & – \\
$\mathcal{L}_{\text {sigmoidF1}}$ & \textbf{0.076} & \textbf{0.103} & \textbf{0.092} & \textbf{0.107} & \textbf{–} \\
\bottomrule
\end{tabular}
\end{table}



\begin{table}
\caption{Arxiv (distillBERT)}

\begin{tabular}{ccccc}
\toprule 
\text { Metric } & $\mathcal{L}_{\text {CE}}$ & $\mathcal{L}_{\text {FL}}$ & $\mathcal{L}_{\text {CE+N}}$ & $\mathcal{L}_{\text {CE+T}} $\\ 
\midrule
 P(\%) & 0 & 0 & 0 & 0 \\ 
R(\%) & 0 & 0 & 0 & 0 \\
$F_{1}$(\%) & 0 & 0 & 0 & \textbf{0} \\
\bottomrule
\end{tabular}
\end{table}

\begin{table}
\caption{Cancer hallmark (distillBERT)}
\centering
\begin{tabular}{ccccc}
\toprule 
\text { Metric } & $\mathcal{L}_{\text {CE}}$ & $\mathcal{L}_{\text {FL}}$ & $\mathcal{L}_{\text {CE+N}}$ & $\mathcal{L}_{\text {CE+T}}$ \\ 
\midrule 
P(\%) & 0 & 0 & 0 & 0 \\ 
R(\%) & 0 & 0 & 0 & 0 \\
$F_{1}$(\%) & 0 & 0 & 0 & \textbf{0} \\
\bottomrule
\end{tabular}
\end{table}

\begin{table}
\caption{Chemical exposure (distillBERT)}
\centering
\begin{tabular}{ccccc}
\toprule 
\text { Metric } & $\mathcal{L}_{\text {CE}}$ & $\mathcal{L}_{\text {FL}}$ & $\mathcal{L}_{\text {CE+N}}$ & $\mathcal{L}_{\text {CE+T}}$ \\ 
\midrule 
P(\%) & 0 & 0 & 0 & 0 \\ 
R(\%) & 0 & 0 & 0 & 0 \\
$F_{1}$(\%) & 0 & 0 & 0 & \textbf{0} \\
\bottomrule
\end{tabular}
\end{table}

\begin{table}
\caption{Chemical exposure (distillBERT)}
\centering
\begin{tabular}{ccccc}
\toprule 
\text { Metric } & $\mathcal{L}_{\text {CE}}$ & $\mathcal{L}_{\text {FL}}$ & $\mathcal{L}_{\text {CE+N}}$ & $\mathcal{L}_{\text {CE+T}}$ \\ 
\midrule 
P(\%) & 0 & 0 & 0 & 0 \\ 
R(\%) & 0 & 0 & 0 & 0 \\
$F_{1}$(\%) & 0 & 0 & 0 & \textbf{0} \\
\hline
\end{tabular}
\end{table}

\begin{table}
\caption{Simulated data}
\centering
\begin{tabular}{ccccc}
\toprule
\text { Metric } & $\mathcal{L}_{\text {CE}}$ & $\mathcal{L}_{\text {FL}}$ & $\mathcal{L}_{\text {CE+N}}$ & $\mathcal{L}_{\text {CE+T}}$ \\ 
\midrule 
P(\%) & 0 & 0 & 0 & 0 \\ 
R(\%) & 0 & 0 & 0 & 0 \\
$F_{1}$(\%) & 0 & 0 & 0 & \textbf{0} \\
\bottomrule
\end{tabular}
\end{table}

%%% Local Variables:
%%% mode: latex
%%% TeX-master: "../main"
%%% End:
