% !TEX root = ../main.tex

\section{Related Work}
\label{sec:org2aceb9f}
\todo{clean-up}


Beyond identifying object types (see YOLO~\cite{YOLO} and its successors), performing face recognition (see FaceNet\cite{FaceNet} and its successors) on segments of an image, neural networks
are increasingly becoming better at predicting more abstact concepts via
deeper networks, representation learning and self-supervision~\citep[see,
e.g.,][]{SS,Rep}. Towards this goal, there is a significant volume of recent
work on building neural networks with a high-level of abstract understanding
in the embedding space~\mdr{REF}. However, research on developing optimization
frameworks that are adapted for these abstract concepts in the output space is
limited.

\subsection{fit-data-to-algorithm}

The \textit{label powerset} approach considers each unique set of labels as one class in the transformed setting~\cite{multilabelComparison} (e.g. an instance labeled \textit{Thriller} and \textit{action}, results in the creation of the class \textit{Thriller and action}). Alternatively, \textit{ranking by pairwise comparison} is a solution where the dataset is duplicated for each possible label pairs. Each duplicated dataset has therefore two classes and only contains instances that have at least one of the labels in the label pair. Different ranking methods exist~\cite{pairwiseBinary, pairwiseNet}. 

More recently, hierarchical datasets such as DBpedia\footnote{https://wiki.dbpedia.org/develop/datasets/latest-core-dataset-releases} are often used to finetune BERT-based models~\cite{XLNet, bigBird}. In both publications, cross-entropy is used to predict the labels.

\subsection{fit-algorithm-to-data}

Early representatives of fit-algorithm-to-data stem from heterogenous
domains of machine learning. Multi-Label k-Nearest Neighbors \cite{ML-KNN},
Multi-Label Decision Tree \cite{ML-DT}, Ranking Support Vector Machine
\cite{multilabelSVM} and Backpropagation for Multi-Label Learning
\cite{multilabelBackprop}. More recently, two papers introduced the idea of
multitask learning for \emph{label prediction} and \emph{label count
prediction} for text (ML\(_{\text{NET}}\)) \cite{multitaskLabel} and image
\cite{multitaskLabelImages, tencent} data. The latter research is loosely catered
towards object detection (although not formally presented as such) and is thus
out-of-scope: elements in a picture are predicted that tend to be unilabel as
defined by the groundtruth (e.g. cat, flower, vase, person, bottle etc.).


\subsection{thresholding}
\label{subsec:thresh}

Machine learning prediction tasks' output are probabilistic (or a reversible transformation of a probabilistic measure such as a sigmoid or a soft max function). At training time, these probabilistic values are compared to binary values in the case of binary encoding of classes. At inference time, if the number $n_i$ of labels to be predicted per example is known a priori, it is natural to assign the $top_{n_i}$ predictions to that example~\cite{lossTopKError, topKmulticlassSVM}. If the number of labels per example is unknown a priori, the question remains at inference time as to how to extract information about the number of labels to assign to each example, aside from the propensity of labels to be assigned. This is generally done via a \emph{decision threshold}, that can be set globally for all examples. This threshold can optimize for specificity or sensitivity~\cite{decisionThreshold}. We propose a method where this threshold is implicitely defined, thanks to the use of metrics that already penalize for wrong label counts.

Thresholding accross classes or examples can be an issue as soon as the number of labels to predict is unknown. Certain variants of cross-entropy loss accommodate imbalanced label data  \cite{focalLoss}, but remain agnostic towards the number of labels to predict. Solutions have been tailored to that end, starting with determining an ideal global \emph{threshold} depending on use-cases \cite{threshForF1}, or per-class-thresholding after training \cite{moviePosters} and eventually abstracting the threshold away via a \emph{soft-F1} measure \cite{softF1}. In the latter two cases, the task is to predict genre from movie posters.

\subsection{Metrics as Losses}

Learning to Rank (the practice of using Machine Learning to sort documents according to their relevance) lead to the widespread use of certain metrics~\cite{LTR}. In a number of retrieval tasks, a model's out of sample accuracy is measured on metrics such as AUROC, F1 score, etc. These reflect an objective catered towards evaluating the model over an entire ranking. Due to the lack of differentiability, these metrics cannot be directly used as loss functions at training time (in-sample). A seminal study~\cite{optimizableLosses} derived a general framework for deriving decomposable surrogates to some of these metrics. We propose our own decomposable surrogates of classical confusion matrix metrics and in particular sigmoidF1 tailored for the problem at hand.

Often, machine learning post-training evaluation metrics (e.g. AUROC, F1) are
not differentiable. There are motivations \todo{which motivations} for
optimizing a model directly on a metric at training time. A general framework
for AUC, AUROC and F1 is presented in \cite{optimizableLosses}, but the
proposed F1 surrogate remains short of being explicitly derived for stochastic
gradient descent. Recently, a similar work has been proposed to train a
Convolutional Neural Network (CNN) from scratch with a few millions of images
and hundreds of labels specifically for multilabel tasks \cite{tencent}. This
task is loosely related to object detection, similarly to
\cite{multitaskLabelImages} mentioned in the previous paragraph.


The proposed method is positioned in the lineage of \emph{algorithm adaptation}, using \emph{metric as losses} and allowing for dynamic \emph{thresholding}. 

% This section will be guided by the previous section's formulation of the multitags problem, we will therefore focus on \emph{fit-algorithm-to-data}, \emph{metrics as losses} and \emph{thresholding}.

% \subsection{fit-algorithm-to-data}
% \label{sec:org150a474}

% % Early representatives of \emph{fit-algorithm-to-data} stem from heterogenous domains of machine learning. Multi-Label k-Nearest Neighbors \cite{ML-KNN}, Multi-Label Decision Tree \cite{ML-DT}, Ranking Support Vector Machine \cite{multilabelSVM} and Backpropagation for Multi-Label Learning \cite{multilabelBackprop}. More recently, two papers introduced the idea of multitask learning for \emph{label prediction} and \emph{label count prediction} for text (ML\(_{\text{NET}}\)) \cite{multitaskLabel} and image \cite{multitaskLabelImages} data. The latter research is loosely catered towards object detection (although not formally presented as such) and is thus out-of-scope: elements in a picture are predicted that tend to be unilabel as defined by the groundtruth (e.g. cat, flower, vase, person, bottle etc.).

% \subsection{Metrics as losses}
% \label{sec:orgb0a9d21}

% Often, machine learning post-training evaluation metrics (e.g. AUROC, F1) are not differentiable. There are motivations \todo{which motivations} for optimizing a model directly on a metric at training time. A general framework for AUC, AUROC and F1 is presented in \cite{optimizableLosses}, but the proposed F1 surrogate remains short of being explicitly derived for stochastic gradient descent. \todo{check again with the authors if I can't get inspired from their work}. Recently, a similar work has been proposed to train a Convolutional Neural Network (CNN) from scratch with a few millions of images and hundreds of labels specifically for multilabel tasks \cite{tencent}. This task is loosely related to object detection, similarly to \cite{multitaskLabelImages} mentioned in the previous paragraph.


% in reformulating loss functions to accomodate sparsity in the data, to optimize directly for the metric at hand or to do thresholding posthoc (see movie posters).

% \subsection{Thresholding}
% \label{sec:org8295f09}

% \emph{thresholding} accross classes or examples can be an issue as soon as the number of labels to predict is unknown. Certain variants of cross-entropy loss accommodate imbalanced label data  \cite{focalLoss}, but remain agnostic towards the number of labels to predict. Solutions have been tailored to that end, starting with determining an ideal global \emph{threshold} depending on use-cases \cite{threshForF1}, or per-class-thresholding after training \cite{moviePosters} and eventually abstracting the threshold away via a \emph{soft-F1} measure \cite{softF1} \todo{say more about this method}. In the latter two cases, the task is to predict genre from movie posters.



% The proposed method is positioned in the lineage of \emph{fit-algorithm-to-data}, using \emph{metric as losses} and allowing for dynamic \emph{thresholding}.

% \todo{compare to this:} \cite{lossComp}

% \todo{Hamming Loss}
% % \todo{Precision@K, Recall@K, NDCG@K}
% \todo{MLTSVM loss and the three-way loss inspired by it} \cite{MLTSVM} and \cite{MLTSVMThreeway}

% We propose a dynamic thresholding mechanism auto-tuned at training time.


% ** weak labels
% (unsure the labels are correct)

% - https://people.cs.pitt.edu/~kovashka/ye_zhang_kovashka_iccv2019_cap2det.pdf


% ** implementations

% *** movies

%  [[https://www.analyticsvidhya.com/blog/2019/04/build-first-multi-label-image-classification-model-python/][movie posters with classes]].

%  They have movie titles in them

% *** pretrained resnet on multilabel

%  https://github.com/Tencent/tencent-ml-images

% What happens when using a Resnet pretrained on multilabels

% *** soft F1 score loss

%  https://github.com/ashrefm/multi-label-soft-f1

% https://www.analyticsvidhya.com/blog/2019/04/build-first-multi-label-image-classification-model-python/



% /Optimizing directly for macro F1: By introducing the macro soft-F1 loss, we could train the model to directly increase the metric we care about: the macro F1-score @ threshold 0.5. We could clearly observe the alignment during training and evaluation on successive epochs. When using this loss, we do not have to tune the decision threshold any more. Imagine a multi-label classification system with hundreds of labels, how unstable the system will be if we have to continuously update the optimal threshold for each label. The macro soft-F1 loss comes to the rescue. By using it, we can keep all thresholds fixed at 0.5 and still get an optimal performance from the training process./



%%% Local Variables:
%%% mode: latex
%%% TeX-master: "../main"
%%% End:
