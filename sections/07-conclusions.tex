% !TEX root = ../main.tex

\section{Conclusion}
\label{sec:orged3d8a1}
% \mdr{Structure the conclusion in five paragraphs, devoted to the following questions:}
% \begin{itemize}[leftmargin=*]
% \item What did we do
% \item What did we find
% \item What are the implications
% \item What are the limitations
% \item What should we do next
% \end{itemize}


\paragraph{Results}
In this paper, we defined a new problem that we call Full-Instance, Multilabel Prediction for Unknown Label count (FIMPUL). It is a subdomain of multiclass classification where an instance needs to be considered as a whole (movie, paper abstract) in order to be labeled. The available labels are from different classes and can be of different numbers per example. This problem is getting addressed more frequently with advances in deep learning providing abstract representation of the input space for multiple modalities that can be used to effectively solve downstream tasks (e.g. with models of the BERT and CNN families). 

To address FIMPUL problems, we proposed a general loss framework for confusion metrics metrics (CoMMaL). We performed  experiments with a specific loss function from the CoMMaL framework, sigmoidF1. We found that sigmoidF1 can achieve significantly better results for most metrics on four diverse datasets and that sigmoidF1 outperforms other losses on the weightedF1 metric.
b
More generally, CoMMaL, our smooth formulation of confusion matrix metrics allows to optimize directly for these metrics that are usually reserved for the evaluation phase. 

\paragraph{Limitations}
We evaluate the CoMMaL framework and sigmoidF1 loss function only on FIMPUL problems. It can be worth to consider the effectiveness of the CoMMaL framework in different domains. More experimentation is also needed to find a proper heuristic for finetuning sigmoidF1's hyperparameters. However, we do point out that the proposed unboundedF1 counterpart does not require tuning and delivered better results than existing multiclass losses on most metrics. It can act as a less mathematically robust substitute of sigmoidF1.

% It is also debatable whether that niche is 
% it is debatable wether any task is intrinsincly multilabel and wether the image / text cannot be decomposed in parts that are single labeled.

% % not long training and small models, but aibility to demonstrate the statement anyways.



\paragraph{Future work}
In future work, we would like to test the limits of CoMMaL and sigmoidF1 both within and beyond the FIMPUL setting.

But first, if provided enough computing power, one could perform the same experiments with micro losses: given enough representatives of each confusion matrix quadrant for each class, one could consider formulating a micro F1 as a loss. The case where the number of classes is small, the dataset is sizeable and enough memory is available (especially with the recent advances in model parallelism) would be favorable to that end.

%% future work %%
% datasets
% labeling setting
% neural architecture
% SOTA transfer learning
% train from scratch

A first step within the FIMPUL setting, could be to use more robust transfer learning / finetuning procedures, for example with dynamic weight freezing for finetuning~\cite{ULMFit}. Alternatively, we would like to implement the smooth losses to train a CNN or a BERT model for FIMPUL tasks from scratch (c.f., \cite{tencent} and \cite{focalLoss}). If training from scratch, it might then be interesting to combine the proposed loss functions with representation learning \cite{unsupervisedImage,highResRepresentation} or self-supervised learning, in order to model abstract relationships between the labels.

Furthermore, we propose to tackle other multilabel settings, such as hierarchical multilabel classification \cite{HARAM},  active learning (for both see \cite{activeLearningMultiLabel}) or \emph{extreme} multilabel prediction \cite{extremeMultilabelText, extremeSIGIR}, where the number of classes range in the tens of thousands. More generally, sigmoidF1 could be tested on any model that use F1 score as an evaluation metric such as AC-SUM-GAN \cite{AC-SUM-GAN}.

The recently emerged \textit{holistic} content labeling tasks might be another promising testing ground for CoMMaL and sigmoidF1. Holistic labeling refers to labels given to an entire content (full-instance) at different levels of abstraction. A dataset was recently released for \emph{Large Scale Holistic Video Understanding}~\cite{holisticVideoData}.\footnote{Available at \url{https://github.com/holistic-video-understanding/HVU-Dataset}}

\vspace{\baselineskip}

In the mean time, if this paper is accepted, we will release our results on Kaggle website for the arXiv multi-label text classification task\footnote{\url{https://www.kaggle.com/Cornell-University/arxiv/tasks?taskId=1757}} in the months to come.




%%% Local Variables:
%%% mode: latex
%%% TeX-master: "../main"
%%% End:
