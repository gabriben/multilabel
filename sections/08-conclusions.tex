% !TEX root = ../main.tex

\section{Conclusion}
\label{sec:orged3d8a1}
\mdr{Structure the conclusion in five paragraphs, devoted to the following questions:}
\begin{itemize}[leftmargin=*]
\item What did we do
\item What did we find
\item What are the implications
\item What are the limitations
\item What should we do next
\end{itemize}


\textbf{Shortcomings}

it is debatable wether any task is intrinsincly multilabel and wether the image / text cannot be decomposed in parts that are single labeled.

not long training and small models, but aibility to demonstrate the statement anyways.

\textbf{Results}

In this paper we defined a new problem in deep learning for mulitple modalities that harness the current advances in abstract representation of the input space. A general loss framework is proposed to locate that solution within the existing multiclass multilabel losses and a specific loss function is formulated. \emph{sigmoidF1} achieves significantly better results for different F1 metrics on all datasets.

\textbf{Future work}

Apply the loss function to more sophisticated neural network architectures that use F1 score as an evaluation metric such as AC-SUM-GAN \cite{AC-SUM-GAN}.

This model can be adapted for hiarchical multilabel classification \cite{HARAM} or active learning (for both see \cite{activeLearningMultiLabel}).

Combine the proposed loss functions with representation learning \cite{unsupervisedImage,highResRepresentation} or self-supervised learning, in order to model abstract relationships between the labels.

adapt to \emph{extreme} multilabel prediction \cite{extremeMultilabelText}

implement the loss to train a CNN from scratch as was done by \cite{tencent}